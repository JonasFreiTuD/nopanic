\chapter{Modulübersicht}

Ein (M) kennzeichnet ein Modul nur für Medieninformatiker, ein (I) jeweils Module für Informatiker und ein (D) für Diplominformatiker.
Die Modulnummern orientieren sich an den der Bachelorstudiengänge.
Für Diplomstudenten können einige dieser Nummern aufgrund andere CreditPoint-Menge anders lauten, da offiziell ein anderes Modul besucht wird.
Z.B. das Modul INF-B-240 Programmierung hat für Diplomstudenten die Nummer INF-D-230.
Das Modul INF-B-380 Betriebssysteme und Sicherheit hat hingegen jedoch die selbe Nummer.

\textbf{1. Semester}

\textbf{INF-B-110 Einführung in die Mathematik für Informatiker (I+M+D)} \\
Ihr kennt euch mit Matrizen aus?
Dann wisst ihr auch was mit den Begriffen Determinante, Diagonalierbarkeit, Skalarprodukt und Lösung eines homogenen linearen Gleichungssystems anzufangen - wenn nicht, dann lernt ihr es hier von der Pike an.
Außerdem wird in der Diskreten Mathematik das Mal und Plus quasi neu definiert und ihr lernt ein wenig anders zu denken.

\textbf{INF-B-210 Algorithmen und Datenstrukturen (I+M+D)} \\
Was kommt zuerst?
5 oder 3?
Solche Fragen werden euch in Algorithmen und Datenstrukturen beschäftigen während ihr Quicksort, Heapsort und Konsorten lernt.
Weiter werdet ihr euch als Gärtner versuchen indem ihr AVL- und andere Bäume wachsen lasst.
Dabei werdet ihr Bekanntschaft mit der Programmiersprache C machen.

\textbf{INF-B-230 Einführungspraktikum (I+M+D)} \\
Ihr habt schon immer gerne mit Lego gespielt?
Dann wird euch dieses Praktikum, welches in der vorlesungsfreien Zeit stattfindet, gefallen.
Ihr dürft euch im Team daran machen einen selbst konstruierten Roboter in C beizubringen, wie er sich in einem Labyrinth alleine zurechtfindet.
Dabei, und im anschließenden Wettbewerb kommt der Spaß nicht zu knapp.
Für Diplomstudenten gibt es stattdessen ein Einzelprojekt, bei dem man zeigen kann, was man in C drauf hat.
Im letzten Jahr war eine KI für bekannte Brettspiele gefragt.

\textbf{INF-B-410 Einführung in die Medieninformatik (I+M)} \\
Anfangs erfolgt eine Darstellung des menschlichen Wahrnehmungssystems, Aspekte der Wahrnehmungspsycholgie und der Softwareergonomie.
Dann werden Eigenschaften der Information und Datenformate anhand der Medien Text, Bild, Audio und Video dargestellt.
Im Bereich Text und Bild werden die entsprechenden Dokumentenformate des Internet (HTML und SVG) besprochen.
Ein weiterer Teil der Lehrveranstaltung gibt einen Überblick zur Dokumentenverarbeitung mittels XML-Techniken.
Die Praxis wird in den Gruppenübungen erworben und in kleinen Teams ein Projekt erarbeitet.

\textbf{INF-D-420 Technische Grundlagen und Hardwarepraktikum (D)} \\
siehe INF-B-390, 3. Semester.

\textbf{INF-B-330 Rechnerarchitektur (D)} \\
siehe INF-B-330, 3. Semester.

\textbf{2. Semester}

\textbf{INF-B-120/INF-D-120 Mathematische Methoden für Informatiker (I+M+D)} \\
Nachdem der Abistoff wieder und viel tiefer als vorher sitzt, geht es in den nächsten zwei Semestern in neue Bereiche der Mathematik.
Anfangs werden die verschiedenen Typen algebraischer Strukturen (das sind Mengen von beliebigen Symbolen und darauf erklärte Rechenoperationen) untersucht.
Es folgen Vektoren, Matrizen und mathematische Körper.
Dann kommt ein Sprung vom Diskreten zum Kontinuierlichen.
So langweilig wie in der Schule ist Analysis nämlich gar nicht, die gibt es auch in der Ausführung mit mehreren Veränderlichen.
Das Ganze gipfelt in der Einführung von Differentialgleichungen.
Gegen Schluss wendet man sich erneut den Polynomen zu.
Dabei werden zunächst effiziente Näherungsverfahren behandelt.
Später folgt dann ein kurzer Ausflug in die Stochastik.

\textbf{INF-B-240/INF-D-230 Programmierung (I+M+D)} \\
Dass Programmiersprachen nicht auf Bäumen wachsen, wusstet ihr wahrscheinlich schon, doch dass sie strengen mathematischen Regeln folgen, lernt ihr hier.
Am Beispiel eines Teils der Programmiersprache C wird zunächst die Syntax mit Hilfe von Grammatiken definiert.
Kurz darauf kommt ihr in den Kontakt mit der funktionalen Programmiersprache Haskell.
Durch viele hübsche, rekursiv verschachtelte Abbildungen wird dann die Semantik festgelegt, d.h. die Wirkung, die so ein Programm auf einer (abstrakten) Rechenmaschine hat.
Hier wird auch vermittelt, wie man die Korrektheit eines Programmstückes "wasserdicht", d.h. formal logisch beweisen kann.

\textbf{INF-B-260/INF-D-310 Informations- und Kodierungstheorie (I+M+D)} \\
Was Informationen eigentlich sind, was sie ausmacht, wird euch hier beschäftigen.
In dieser Lehrveranstaltung werdet ihr einen Einstieg in ein sehr interessantes und komplexes Fachgebiet erhalten.
Im Mittelpunkt steht am Anfang wie man Informationen darstellen und speichern kann.
Etwas später wird erklärt, warum und wie die Informationen mittels Kodierung geschützt werden, damit sie bei euch sicher ankommen, wenn sie unterwegs Störungen und Manipulationen ausgesetzt sind.
Dabei wird euch euer in der Mathematik erworbenes Wissen von Nutzen sein.

\textbf{INF-B-310/INF-D-240 Softwaretechnologie (I+M+D)} \\
Software zu entwickeln ist eine Kunst, das werdet ihr spätestens nach diesem Modul erkennen.
Um diese Kunstfertigkeit an den Tag legen zu können bedarf es einiger Handwerkszeuge, welche ihr hier mit auf den Weg bekommt.
So werden euch moderne Konzepte am Beispiel von Java und Entwurfsverfahren zusammen mit professioneller Dokumentation näher gebracht.
Damit wird dann der Grundstein für das Projekt im dritten Semester gelegt, bei dem man sich Lorbeeren im Projektmanagement und als Entwickler verdienen kann.

\textbf{INF-B-420 Einführung in die Computergraphik (I+M)} \\
Es geht um den Aufbau von Grafiksystemen, Farbräumen, Rastergraphiken und deren Anwendungen.
Bestehende Probleme, wie Aliasing und Artefakte sind mit von der Partie, sowie ihre algorithmischen Lösungen.
Als Programmiersprache für die Übungsaufgaben wird C++ genutzt.

\textbf{INF-D-420 Technische Grundlagen und Hardwarepraktikum (D)} \\
Fortsetzung aus dem 1. Semester.

\textbf{INF-B-330 Rechnerarchitektur (D)} \\
Fortsetzung aus dem 1. Semester.

\textbf{3. Semester}

\textbf{INF-B-120/INF-D-120 Mathematische Methoden für Informatiker (I+M+D)} \\
Fortsetzung aus dem 2. Semester.

\textbf{INF-B-270 Formale Systeme (I+M+D)} \\
Wahr?
Und oder falsch?
Was falsch ist, wird, wenn es falsch falsch ist, wahr?
Logisch!
Neben der Aussagenlogik vermittelt das Modul die Grundlagen formaler Sprachen.
Es folgen Gedanken zu maschinellen Berechenbarkeit und zur Automatentheorie.
Turing lässt grüßen.

\textbf{INF-B-230 Softwaretechnologie-Projekt (I+M+D)} \\
Das Projekt nimmt den größten Teil des dritten Semesters ein.
Hier muss man sein Wissen aus der Lehrveranstaltung "Softwaretechnologie" in die Tat umsetzen.
In einem fünfköpfigen Team hat man die Aufgabe, eine Anwendung von vorn bis hinten fertig zu stellen.
Dabei muss man häufig Rücksprache mit den "Kunden" halten.
Abgeschlossen wird das Modul mit einer Präsentation des fertigen Produkts vor dem Kunden und den Verantwortlichen des Moduls.
Am Ende habt ihr dann einen Eindruck, wie die Arbeit eines Informatikers aussehen kann.

\textbf{INF-B-330 Rechnerarchitektur (I+M)} \\
Hier geht es um die Grundbausteine eines Computers:
Speicher, Bussysteme, Rechen- und Steuerwerk.
Außerdem erhält man eine Einführung in Assembler, das Pipelining-Prinzip und damit auftretende Probleme.
Schließlich wird noch diskutiert, mit welchen Methoden man heutige Rechnerarchitekturen beschleunigen kann und parallele Architekturen nutzen kann.

\textbf{INF-B-390 Technische Grundlagen und Hardwarepraktikum (I)} \\
Wer schon immer mal wissen wollte, was die Strömlinge im häuslichen Rechner eigentlich so alles durchmachen müssen, bekommt das genau vermittelt.
Anfangs werden Transistor-, Dioden- und Operationsverstärkerschaltungen betrachtet.
Darauf aufbauend geht es über Verknüpfungsglieder und komplexe Schaltungen.

\textbf{INF-B-440 Grundlagen der Gestaltung (M)} \\
Die Vorlesung beginnt mit Begriffsdefinitionen sowie allgemeinen Gestaltungsprinzipien und erläutert diese.
Dabei beschränkt sich die Veranstaltung bewusst auf zweidimensionale Bereiche.
Formkategorien, Kontrastbildung und Farblehre bilden die Schwerpunkte.
Die begleitenden Übungen sollen einen Einblick in die Materie vermitteln und die Sensibilität der Studierenden durch handwerkliches Arbeiten wecken.

\textbf{INF-D-510 Grundlagen des Nebenfachs (D)} \\
Je nachdem, was ihr euch als Nebenfach wählt, beschäftigt ihr euch hier mit Themen die nur im entfernten Sinne mit Informatik zusammenhängen.
Über den Tellerrand schauen und andere Welten kennenlernen ist das Motto.

\textbf{INF-B-380 Betriebssysteme und Sicherheit (D)} \\
siehe INF-B-380, 5. Semester.

\textbf{4. Semester}

\textbf{INF-B-290/INF-D-330 Theoretische Informatik und Logik (I+D)} \\
Die Fortsetzung der Formalen Systeme.
Es folgen weitere Betrachtung zur Korrektheit und Terminierung von Algorithmen und der notwendige Aufwand in Form von Zeit und Platzbedarf.
Ein Abstecher in die Prädikatenlogik und Logikprogrammierung rundet das Modul ab.

\textbf{INF-B-330 Rechnerarchitektur (I+M)} \\
Fortsetzung aus dem 3. Semester.

\textbf{INF-B-370/INF-D-270 Datenbanken und Rechnernetze (I+M+D)} \\
In dieser Lehrveranstaltung lernt man zuerst Methoden zur effizienten Datenspeicherung kennen.
Danach wird die Fähigkeit vermittelt, selbst komplexe relationale Datenbanken zu konzipieren und zu erstellen.
Auch werden Rechnernetze behandelt.
Angefangen mit dem Funktionsprinzip von Modem und Netzwerk- karte erhält man einen kurzen Überblick über moderne Kommunikations- und Vermittlungsprotokolle.
Auch der Sektor Mobilkommunikation und die dabei auftretenden Schwierigkeiten werden kurz beleuchtet.

\textbf{INF-B-390 Technische Grundlagen und Hardwarepraktikum (I)} \\
Fortsetzung aus dem 3. Semester.

\textbf{INF-B-450 Einführung in die Mediengestaltung (M)} \\
Die Vorlesung vermittelt die Grundzüge des multimedialen Gestaltens unter Gesichtspunkten der Entwicklung der einzelnen Richtungen (Film, Internet) mit Bezug auf die gestalterischen Änderungen in den vergangenen Jahrhunderten (Buch).
Außerdem wird in die Metaphernbildung eingeführt und einige Gastdozenten aus der Praxis vermitteln ihre Sicht auf die Mediengestaltung.

\textbf{INF-B-460 Medien und Medienströme (M)} \\
Hier wird Wissen zu Medien, deren Kompression und Bearbeitung vermittelt.
Die Anwendung verschiedener Werkzeuge zur Erzeugung von Medien und deren Charakteristika sind ebenfalls Gegenstand dieser Lehrveranstaltung.

\textbf{INF-B-470 Medienpsychologie und -Didaktik (M)}
Mediendidaktik ist die "Kunst des Lehrens".
Hier werden die Fragen beantwortet:
Was ist Bildung?
Wie verläuft sie?
Wie lässt sie sich vervollkommnen?
Man erfährt etwas über die Entwicklung von Lehrmethoden.
Im parallel stattfindenden Praktikum wird das Gelernte gleich praktisch bei der Erstellung eines Lernprogramms angewandt.

\textbf{INF-B-490 Komplexpraktikum (M)} \\
Das große Highlight für Medieninformatiker im Bachelor.
In kleineren Gruppen soll eine Broschüre, eine Internet-Seite, einen Film oder Multimediales realisiert werden.
Abgesehen von der Aufgabenstellung sind der Fantasie quasi keine Grenzen gesetzt.
Es geht um harte Arbeit, Teamgeist und das Ernten der wohlverdienten Lorbeeren.

\textbf{INF-D-910 Forschungslinie (D)} \\
Hier bekommt ihr einen Überblick über aktuelle Forschungsthemen und bekommt vermittelt, wie man forschungsorientiert arbeitet.
Dieses Modul hilft, später die richtige Vertiefung zu wählen.

\textbf{INF-D-510 Grundlagen des Nebenfachs (D)} \\
Fortsetzung aus dem 3. Semester.

\textbf{INF-D-510 Allg. Basisqualifikationen (D)} \\
Englisch ist die einzig relevante Sprache in der Informatik.
Hier wird euch vermittelt, wie man sich fachlich auf Englisch ausdrückt.
Abgerundet wird das Modul durch eine Schulung eurer Vortragsfähigkeiten.

\textbf{5. Semester}

\textbf{INF-B-380 Betriebssysteme und Sicherheit (I+M)} \\
Diese Lehrveranstaltung nimmt die dienstbaren Geister, die zwischen der Hardware und den bunten Anwendungen werkeln, unter die Lupe.
Warum kann man mit einem Rechner gleichzeitig einen Text schreiben, compilieren, ein Bild berechnen und Musik hören?
Wie werden meine Daten in Rechnersystemen geschützt?
Wieso stehen die hier auf dieses Unix?

\textbf{INF-B-3A0/INF-D-430 Systemorientierte Informatik/ Hardware Software Codesign (I+D)} \\
Dieses Fachgebiet ist die Schnittstelle zwischen Rechnern und der industriellen Praxis, die von der Steuerung von Heizventilen bis zu Kraftwerken reicht.
Zunächst wird abstrahiert, was allen praktisch vorkommenden Systemen gemein ist, und es werden Modelle wie "System", "Signal" und "Regelkreis" erschaffen, mit denen sich dann rechnerisch umgehen lässt.
Hier wird man fit gemacht für die Analyse und Voraussage von Übertragungsverhalten und Reaktionen, die ein solches System bei einem bestimmten Input zeigen wird.
Daneben kommen auch Aspekte aus der Audio- und Videotechnik wie Digitalisierung und Filteralgorithmen nicht zu kurz.

\textbf{INF-B-3B0/INF-D-340 Intelligente Systeme (I+D)} \\
In dieser Lehrveranstaltung geht es um künstliche Intelligenz.
Hier erlernt man Problemlösung, Wissensrepräsentation, Planung, Wahrnehmung und Sprachverstehen, mit Hilfe spezieller Algorithmen und Agenten.

\textbf{INF-B-480 Web- und Multimedia Engineering (M)} \\
Wie kann man Web mit heutiger Technik multimedial und interaktiv gestalten?
Wie nutze ich professionelle Entwicklungswerkzeuge und geeignete Sprachen, wie z.B. Java, um meine Vorstellung in das Ergebnis zu projizieren?
Dieses Modul hilft geeignete Methoden zu erlernen und Erfahrung bei der Anwendung zu sammeln.

\textbf{INF-B-490 Komplexpraktikum (M)} \\
Fortsetzung aus dem 4. Semester.

\textbf{INF-B-510/530 Vertiefung (I+M+D)} \\
Hier kann der Student aus einem Angebotskatalog geeignete Veranstaltungen wählen um sein wissenschaftlichen Horizont zu vertiefen.
Die Möglichkeiten umfassen Vorlesungen, Übungen, Praktika, Projektbearbeitungen, Exkursionen, Proseminare, Tutorien und Sprachkurse.

\textbf{INF-D-920 Vertiefung im Nebenfach (D)} \\
Nachdem ihr euch die Grundlagen eures gewählten Nebenfachs angeeignet habt, wird es nun ernst und ihr steigt tiefer in die Materie ein.

\textbf{INF-BAS* Basismodul 1, 2 und 3 (D)} \\
Hier wählt ihr unter sieben verschiedenen Themenkomplexen drei aus und beschäftigt euch mit ihnen.
Zur Wahl stehen Angewandte Informatik, Künstliche Intelligenz, Software- und Web-Engineering, Systemarchitektur, Technische Informatik, Theoretische Informatik, Graphische Datenverarbeitung.
Innerhalb dieser Richtungen stehen euch verschiedene Vorlesungen zur Auswahl.
Für mehr Infos müsst ihr die einschlägigen Webseiten und die Prüfungsordnung lesen.

\textbf{6. Semester}

\textbf{INF-B-520/540 Vertiefung zur Bachelorarbeit (I+M)} \\
Weitere Vertiefung nach gleichem Muster wie im fünften Semester in Vorbereitung auf die Bachelorarbeit.

\textbf{INF-B-610 Allgemeine Qualifikation (I+M)} \\
In dieser Art Nebenfach orientiert sich der Student fächerübergreifend an Themen seines Interesse, um die fachspezifische Kompetenz zu entwickeln.
Auch hier können Veranstaltungen aus einem Katalog gewählt werden.

\textbf{Bachelorarbeit und Kolloquium (I+M)} \\
Als krönenden Abschluss fertigt ihr die Bachelorarbeit zu einem von euch gewählten Thema an und verteidigt sie in einem Vortrag.

\textbf{INF-D-920 Vertiefung im Nebenfach (D)} \\
Fortsetzung aus dem 5. Semester.

\textbf{INF-BAS* Basismodul 1, 2 und 3 (D)} \\
Fortsetzung aus dem 5. Semester.

\textbf{7. Semester}

Angehende Diplominformatiker haben nach den sechs Semestern noch vier weitere vor sich.
Im siebten Semester werdet ihr ein Berufspraktikum absolvieren, im achten und neunten werdet ihr dann Module auswählen die euch interessieren und tiefer in die Abgründe des gewählten Themas hinabsteigen.
Im zehnten Semester wird ausschließlich die Diplomarbeit angefertigt und das war es dann schon!
So schnell kann es gehen.