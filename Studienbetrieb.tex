\section{Der Studienbetrieb}

\textbf{Die Grundbegriffe des Studiums in kurzen Worten erklärt.}

Wer "frisch" aus der Schule kommt, kennt als Lehrform vor allem den Dialog. Üblicherweise versucht der Lehrer in der Schule, auf die Denkweise und das Arbeitstempo der Schüler einzugehen, unterhält sich mehr mit ihnen, als dass er ihnen einen Vortrag hält. Am Ende der Stunde hat zumindest ein großer Teil der Schüler den Stoff verstanden. An der Uni gibt es diese Lehrmethode nicht - dafür aber einige andere, an die man sich auch gewöhnen kann. Hier wird viel Wert auf Eigenständigkeit gelegt, ein "an die Hand genommen werden" wie in der Schule, gibt es nicht mehr. Das ist nicht der einzige Unterschied zwischen Schule und Universität. Doch seht selbst:

\textbf{Der Stundenplan}