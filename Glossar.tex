\section{Glossar}

\textbf{AG DSN} \\
Die AG Dresdner Studentennetz kümmert sich um das Internet in einigen Wohnheimen.
Administratoren werden laufend gesucht.
Mehr Infos unter \footnote{}.

\textbf{Anmelden} \\
Alle, die in Dresden heimisch geworden sind, sollten nicht vergessen, sich beim Ortsamt des jeweiligen Stadtbezirkes innerhalb von zwei Wochen anzumelden.
Wo sich das zuständige Ortsamt befindet, kannst du hier\footnote{} nachschlagen.

\textbf{AQuA} \\
Abkürzung für Allgemeine Qualifikation.
Ist ein Bestandteil deines Studiums.
Genaueres:
Siehe Prüfungs- und Studienordnung.

\textbf{Assistent} \\
Wissenschaftlicher Mitarbeiter am Lehrstuhl, meist Doktor.
Leitet oft Übungen oder Seminare.

\textbf{Auslandsstudium} \\
Etwas, das sich im Lebenslauf immer ganz gut macht, von der Erfahrung und möglicherweise guten Bräune ganz abgesehen.
Nähere Informationen gibt es entweder bei uns im FSR oder im Akademischen Auslandsamt\footnote{}.

\textbf{Bachelor} \\
Die neuen bundesweit eingeführten Abschlüsse.
Merkmale sind ein im Vergleich zum Diplom kürzeres Studium und die Möglichkeit, aufbauend einen Master zu erwerben.

\textbf{BAföG} \\
Zum Thema BAföG gibt es sowohl beim StuRa als auch im Studentenwerk Infomaterial und Anträge.
Beantragt wird BAföG beim BAföG-Amt im Studentenwerk\footnote{}, Fritz-Löffler-Str. 18.
Kümmere dich so schnell wie möglich darum, da frühestens ab dem Antragsmonat gezahlt wird.

\textbf{Belegen} \\
Das Hören einer Vorlesung wird auch als Belegen bezeichnet.
Die im Semester gehörten Vorlesungen müssen in den Belegbogen auf der Rückseite des Studienbuchblattes, das dir mit dem Studentenausweis zugeschickt wurde, eingetragen werden.
Dieses solltest du im Studienbuch abheften.

\textbf{Beurlaubung} \\
Auf Antrag gewährt die Uni zwei freie Urlaubssemester.
Nutz diese Möglichkeit, falls Du mal ein Semester frei nehmen willst/musst, damit dir dieses Semester nicht als Fachsemester angerechnet wird.
Achte jedoch auf Bestimmungen zur Höchststudiendauer vor allem zum BAföG.

\textbf{Bibliothek} \\
Primär von Interesse ist für dich die Universitätsbibliothek (SLUB), die du kostenlos nutzen kannst.
Abgesehen davon hast du die Möglichkeit, die städtischen Bibliotheken Dresdens\footnote{} zu nutzen.
Allerdings gibt es für diese eine Jahresgebühr von 12 EUR.

\textbf{Bücher} \\
Es ist ratsam, nicht direkt zum ersten Semester einen Stapel Bücher zu kaufen.
Besser ist es, sich bei höheren Semestern vorher zu erkundigen, welche Literatur ratsam ist.
Außerdem sollte man sich die Bücher, die von Professoren vorgeschlagen werden, zunächst erstmal in der Bibliothek anschauen.
Angebote für gebrauchte Bücher findest du unter anderem in den Campuszeitungen.

\textbf{Campus} \\
Kerngelände der Uni.

\textbf{Campuszeitung} \\
Die zwei Dresdner Campuszeitungen \textit{ad-rem} und \textit{CAZ} erscheinen ein- bzw. zweiwöchentlich.

\textbf{Club Mate} \\
