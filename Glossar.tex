\section{Glossar}

\textbf{AG DSN} \\
Die AG Dresdner Studentennetz kümmert sich um das Internet in einigen Wohnheimen.
Administratoren werden laufend gesucht.
Mehr Infos unter \footnote{}.

\textbf{Anmelden} \\
Alle, die in Dresden heimisch geworden sind, sollten nicht vergessen, sich beim Ortsamt des jeweiligen Stadtbezirkes innerhalb von zwei Wochen anzumelden.
Wo sich das zuständige Ortsamt befindet, kannst du hier\footnote{} nachschlagen.

\textbf{AQuA} \\
Abkürzung für Allgemeine Qualifikation.
Ist ein Bestandteil deines Studiums.
Genaueres:
Siehe Prüfungs- und Studienordnung.

\textbf{Assistent} \\
Wissenschaftlicher Mitarbeiter am Lehrstuhl, meist Doktor.
Leitet oft Übungen oder Seminare.

\textbf{Auslandsstudium} \\
Etwas, das sich im Lebenslauf immer ganz gut macht, von der Erfahrung und möglicherweise guten Bräune ganz abgesehen.
Nähere Informationen gibt es entweder bei uns im FSR oder im Akademischen Auslandsamt\footnote{}.

\textbf{Bachelor} \\
Die neuen bundesweit eingeführten Abschlüsse.
Merkmale sind ein im Vergleich zum Diplom kürzeres Studium und die Möglichkeit, aufbauend einen Master zu erwerben.

\textbf{BAföG} \\
Zum Thema BAföG gibt es sowohl beim StuRa als auch im Studentenwerk Infomaterial und Anträge.
Beantragt wird BAföG beim BAföG-Amt im Studentenwerk\footnote{}, Fritz-Löffler-Str. 18.
Kümmere dich so schnell wie möglich darum, da frühestens ab dem Antragsmonat gezahlt wird.

\textbf{Belegen} \\
Das Hören einer Vorlesung wird auch als Belegen bezeichnet.
Die im Semester gehörten Vorlesungen müssen in den Belegbogen auf der Rückseite des Studienbuchblattes, das dir mit dem Studentenausweis zugeschickt wurde, eingetragen werden.
Dieses solltest du im Studienbuch abheften.

\textbf{Beurlaubung} \\
Auf Antrag gewährt die Uni zwei freie Urlaubssemester.
Nutz diese Möglichkeit, falls Du mal ein Semester frei nehmen willst/musst, damit dir dieses Semester nicht als Fachsemester angerechnet wird.
Achte jedoch auf Bestimmungen zur Höchststudiendauer vor allem zum BAföG.

\textbf{Bibliothek} \\
Primär von Interesse ist für dich die Universitätsbibliothek (SLUB), die du kostenlos nutzen kannst.
Abgesehen davon hast du die Möglichkeit, die städtischen Bibliotheken Dresdens\footnote{} zu nutzen.
Allerdings gibt es für diese eine Jahresgebühr von 12 EUR.

\textbf{Bücher} \\
Es ist ratsam, nicht direkt zum ersten Semester einen Stapel Bücher zu kaufen.
Besser ist es, sich bei höheren Semestern vorher zu erkundigen, welche Literatur ratsam ist.
Außerdem sollte man sich die Bücher, die von Professoren vorgeschlagen werden, zunächst erstmal in der Bibliothek anschauen.
Angebote für gebrauchte Bücher findest du unter anderem in den Campuszeitungen.

\textbf{Campus} \\
Kerngelände der Uni.

\textbf{Campuszeitung} \\
Die zwei Dresdner Campuszeitungen \textit{ad-rem} und \textit{CAZ} erscheinen ein- bzw. zweiwöchentlich.

\textbf{Club Mate} \\
Das ultimative Kultgetränk unter Hackern dieser Welt und im ASCII erhältlich.
Positiver Nebeneffekt nach dem Genuss von Club Mate ist, dass der hohe Koffeingehalt munter macht/hält.
(Nicht mehr ganz so Geheim)Tipp:
Auch mal die Dresdner Kolle-Mate im ASCII probieren.

\textbf{Creditpoints} \\
Sammelst Du mit dem Bestehen von Modulen.
Die Anzahl gibt an wieviel Zeit Du aufgewendet hast, bzw. haben sollst.

\textbf{DAAD (Deutscher Akademischer Austauschdienst)} \\
Deutschlandweite Anlaufstelle für das Auslandsstudium\footnote{}.

\textbf{Dekan} \\
Der Dekan leitet und vertritt die Fakultät und führt die Beschlüsse des Fakultätsrates aus.
Der gegenwärtige Dekan ist Prof. Baader.

\textbf{dies academicus} \\
Am "akademischen Tag" finden anstelle der Vorlesungen und Übungen andere Veranstaltungen statt.
Er dient dazu, den Studenten die Möglichkeit zu geben, einmal einen Blick in andere Fachbereiche zu werfen.
Häufig veranstalten die verschiedenen Fachschaftsräte auch Sportturniere an diesem Tag.

\textbf{Diplom} \\
Alternativer Studienabschluss zum Bachelor.
Im Wintersemester 2010 wurde ein neuer, modularisierter Diplomstudiengang (nur Informatik und Informationssystemtechnik) an unserer Fakultät eingeführt.
Im Gegensatz zum Bachelor bietet dieser ein Nebenfach und ein Praktikumssemester.
Das Diplom berechtigt zur Promotion zum Doktor.

\textbf{DrePunct} \\
Bibliothek am Zelleschen Weg 17, die unter anderem die Bücher des Fachbereichs Informatik beinhaltet.

\textbf{Emeal} \\
Der Emeal (auch Mensakarte) wird gebraucht, um in den meisten Mensen Essen zu bekommen.
Er ist gegen eine Kaution von 5 EUR und Vorlage der Emeal-Bescheinigung sowie des Personalausweises an den Kassen der Mensen erhältlich.
Zu Beginn des jeweils nächsten Semesters muss der Emeal verlängert werden.
Im Rahmen der ESE wird die Mensakarte aber auch direkt ausgeteilt.

\textbf{Erasmus} \\
Eine europaweite Initiative zum Studentenaustausch\footnote{}.
Siehe auch Auslandsstudium.

\textbf{EVA} \\
Lehrevaluation, gegen Ende des Semesters füllst Du in jeder Vorlesung einen Fragebogen aus, um den Dozenten, die Vorlesung und die Übungsleiter zu bewerten.

\texbt{Exmatrikulation} \\
Beim Austritt aus der Hochschule (Studienende/-abbruch, Wechsel der Hochschule) muss man sich exmatrikulieren.
Zwangsweise geschieht dies, wenn man die Höchststudiendauer überschreitet oder vergisst, sich rückzumelden oder notwendige Prüfungen endgültig nicht bestanden hat.

\textbf{Fachschaftsrat} \\
Gewählte studentische Vertreter einer Fachschaft.
Deine studentischen Vertreter findest Du im Raum E017 oder online\footnote{ifsr.de}.
Der Fachschaftsrat freut sich auch immer über Studenten, die mal vorbeischauen und über Probleme oder Anregungen berichten.

\textbf{Fachschaft} \\
Alle Studenten einer Fakultät. Also auch Du.

\textbf{Fachschaftsratsitzung} \\
Findet einmal wöchentlich im Fachschaftsrat statt.
Hier werden Aktionen geplant, Angelegenheiten der Fakultät diskutiert und vieles mehr.
Jeder ist dazu herzlich eingeladen!
Termine und Sitzungsprotokolle gibt es auf der FSR-Homepage\footnote{ifsr.de}.
Derzeit:
Jeden Montag 18.30 Uhr im großen Ratssaal (INF/1004).

\textbf{Fakultät} \\
In Fakultäten werden verschiedene Fachrichtungen zu einer Lehr- und Verwaltungseinheit zusammengeschlossen (z.B. Fakultät Informatik, Philosophische Fakultät, etc.).

\textbf{FFFI} \\
Der Förderverein "Freunde und Förderer der Informatik der TU-Dresden e.V.".
Mehr Information auf der Webseite\footnote{}.

\textbf{FRZ (ZIH)} \\
Das Rechenzentrum in der Informatikfakultät wurde früher von dieser betrieben.
Heute gehört es mit zum ZIH.
Der Rechnerpool bietet Dir Gelegenheit, deine Projekte innerhalb der Fakultät zu bearbeiten.
Vorlesungsskripte und Übungsaufgaben einsehen und ausdrucken gehört zu den häufigeren Nutzungen der Rechner.

\textbf{Hochschulsport} \\
Siehe USZ.

\textbf{Immatrikulationsamt} \\
Zuständig für Aktivitäten wie Immatrikulation, Exmatrikulation und Rückmeldung.
Zu finden im Bürohaus Strehlenerstr. 24, 6. Etage und im Netz\footnote{}.

\textbf{I'm So Meta Even This Acronym} \\
%oh yeah

