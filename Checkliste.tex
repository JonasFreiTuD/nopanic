\section{Erstsemester-Checkliste}

Für einen erfolgreichen Start in das Studium solltest du einige organisatorische Kleinigkeiten unbedingt in den ersten Wochen erledigen.
Diese haben wir dir in folgender Checkliste zusammengestellt.
Die "ToDos" sind in absteigender Priorität geordnet, d.h. je weiter oben etwas in der Liste steht, desto dringender solltest du dich darum kümmern.

%TODO: genaues Datum hier noch einfügen
\textbf{Bis zum Ende der ESE Woche}

Wohnung \\
Solltest du noch keine Bleibe gefunden haben, ist Beeilung angesagt, die schönsten Wohnungen sind schnell weg.
Wenn du in den Genuss eines 10- bzw. 100-Mbit/s-Internetzugangs kommen möchtest, seien dir die Wohnheime \footnote{} des Studentenwerks Dresden empfohlen.

Studienrelevante Dokumente \\
Das Vorlesungsverzeichnis und die Prüfungs- und Studienordnung erhälst du direkt beim Prüfungsamt \footnote{}.
Gedruckte Ordnungen gibts beim FSR und in deiner ESE-Tüte.
Alle wichtigen Informationen zu den einzelnen Vorlesungen findest du auf den jeweiligen Seiten der Institute im Netz.
Die Professoren werden dir dazu jedoch auch noch alles in den ersten Vorlesungen mitteilen. Sonst hilft natürlich schon einmal ein Blick auf die Seite des FSR \footnote{}.

Mail Account \\
Siehe ZIH HowTo in diesem Heft.

E-Meal Karte \\
Die Mensa Karte gibt es während der ESE oder in den Mensen selbst für jeweils 5 EUR Pfand.
Zusätzlich dazu benötigst du eine E-Meal Bescheinigung, die du auf deinem Semesterbogen findest.

(optional) Sprachkurse \\
Die Einschreibung für die Sprachkurse wird je nach Kurs im Lauf der ersten beiden Wochen deines Studiums freigeschaltet.
Erkundige dich auf den Seiten des LSK \footnote{} frühzeitig, wann dies ist. Die meisten Kurse sind sehr schnell voll.

(optional) Sportkurse \\
Wie für die Sprachkurse gilt auch hier, wer zuerst da ist...
Das Angebot könnt ihr beim Universitätssportzentrum (USZ) einsehen \footnote{}.
Habt ihr euch für einen Kurs entschieden und bei freigeschalteter Einschreibung für diesen angemeldet, müsst ihr nur noch die Anmeldebescheinigung drucken und den Kostenbeitrag innerhalb von drei Tagen auf das Konto des USZ überweisen.

\textbf{Bis Ende Oktober}

Wohnsitz anmelden \\
Offiziell musst du innerhalb von zwei Wochen entweder beim Studentenwerk oder beim zuständigen Ortsamt \footnote{} deine Wohnung anmelden.
Wer seinen Hauptwohnsitz nach Dresden verlegt bekommt von der Stadt eine "Umzugsbeihilfe" in Höhe von 150 EUR.
Informationen dazu gibt's unter \footnote{} und \footnote{}.
Beachte ebenfalls, dass du in den meisten Fällen bei einer Anmeldung deiner Bleibe als Nebenwohnung keine Zweitwohnungssteuer mehr zahlen musst!
Sollte dennoch ein Steuerbescheid der Stadt kommen, musst du diesem innerhalb eines Monats widersprechen.
Berufe dich dabei auf das Verfahren mit dem Aktenzeichen 2 K 142/07, 2K 141/07 und 140/07 des Verwaltungsgerichtes Dresden aus dem Juli 2007.
Weitere Hilfen zur Begründung findest du beim StuRa \footnote{}.

BAföG Antrag \\
Formulare und Auskunft gibt es beim Studentenwerk (4. Etage).
Schiebe den Antrag nicht allzu lang vor dir her, da dein Anspruch für abgelaufene Monate verfällt.
Informationen zu den Sprechzeiten beim Studentenwerk gibt es hier \footnote{}.

Bibliotheksausweis \\
Bekommt man direkt am Schalter in der SLUB (Zellescher Weg 18) \footnote{}.

\textbf{Weiteres}

Copycard \\
Drucker der Firma Ricoh stehen quer über den Campus verteilt und lassen sich von jedem Rechner mit einer Copycard ansprechen.
Diese bekommst du in der StuRa Baracke hinter dem Hörsaalzentrum für 5 EUR Pfand. Du kannst aber auch direkt beim FSR für 2 Cent/Seite drucken (einfach Dokumente per USB Stick mitbringen).

%TODO: Bei Fredo zwecks Terminplanung anfragen
C und Java-Kurs \\
Besonders denjenigen ohne Programmiererfahrung werden die Wintersemester angebotenen C und Java-Kurse ans Herz gelegt.
Diese finden regelmäßig an Wochenenden statt.
Für Details wendet euch an \footnote{programmierung@ifsr.de oder fredo@ifsr.de} und behaltet die News auf \footnote{ifsr.de} im Auge.

Fachschaftsratwahlen \\
Wähle deine studentischen Vertreter im FSR Informatik.
Die Wahlen finden jedes Jahr im November statt.
Geh wählen!
Und noch besser: Lass dich wählen!

Prüfungseinschreibung \\
Ab Ende Januar kann man sich in jExam zu den Prüfungen anmelden.
Schreib dich in die Prüfungen der Fächer ein, die du besucht hast.
Viel Erfolg!

Rückmeldung zum Sommersemester \\
Ab Mitte Januar 2015 kannst du den Semesterbeitrag für das nächste Semester überweisen.
Den genauen Betrag und Termine findest du auf dem aktuellen Semesterbogen und hier \footnote{tu-dresden.de/studium/organisation/rueckmeldung/semesterrueckmeldung}.
Kümmere dich rechtzeitig darum, sonst wirst du automatisch exmatrikuliert!