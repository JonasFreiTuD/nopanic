\addchap{Press F1 for help}

\minisec{Fachschaftsrat}
Nach deinen Freunden und Seminargruppenmentoren deine nächste Anlaufstelle.
Wir kümmern uns um eure Probleme oder vermitteln Hilfe.

\minisec{Studiendekan}
Neben dem Dekan der Fakultät und seinem Stellvertreter, dem Prodekan, gibt es noch ein weiteres Amt innerhalb der Fakultätsleitung:
den sogenannten Studiendekan.
Er ist für die Angelegenheiten der Lehre in der Fakultät zuständig, bildet den Vermittler zwischen Studenten und Professoren und hilft bei Problemen mit dem Studium allgemein.

Prof. Dr. rer. nat. habil. Weber \\
Büro: INF 1055 \\
Telefon: (0351) 463-38477 \\
E-Mail: gerhard.weber@tu-dresden.de \\

\minisec{Serviceleistungen des Studentenrates (StuRa)}
\begin{itemize}
\item BAföG- und Sozialberatung
\item Rechtsberatung
\item Ausländerberatung
\item Beratung für Studierende mit Kind
\item Beratung zu Anträgen und Förderungsmöglichkeiten
\item Verkauf von Karten für verschiedene Kulturveranstaltungen
\item Material- und Geräteverleih
\end{itemize}

Informationen zu allen Serviceleistungen gibt es im Studentischen Ratgeber \textit{spiritus rector} \link{http://spirex.de} und unter \link{http://www.stura.tu-dresden.de}.

\minisec{Studienberatung}
Möchtest du dich zu deinem Studiengang beraten lassen oder hast Fragen, dann kannst du dich auch gerne an die Studienberatung wenden.
Studentische Berater sind derzeit Sascha Peukert (Informatik) und Philipp Heisig (Medieninformatik).
Erreichbar sind sie unter \textit{studienberatung-inf@ifsr.de} bzw. \textit{studienberatung-minf@ifsr.de} oder gemeinsam unter \textit{studienberatung@ifsr.de}.
Beim Immatrikulationsamt findest du die nicht-studentischen Studienfachberater \link{http://tu-dresden.de/inf/sfb}.

\minisec{Studium mit Behinderung und chronischer Krankheit}
Unter \link{http://tu-dresden.de/inf/bfsb} findet ihr Hilfe und Informationen, um mit Handicap im Studium gut zurecht zu kommen.

\minisec{Prüfungsamt}
Bei Problemen mit der Prüfungseinschreibung, Notenvergabe oder jeglichen Dingen, die mit deinen Prüfungsleistungen zu tun haben ist das Prüfungsamt der entscheidende Ansprechpartner.

Bei Fristüberschreitungen gelten Prüfungen als nicht bestanden und du wirst exmatrikuliert.
Unter Umständen bist du aber gar nicht schuld am Verstreichen eines Termins.
Dann solltest du einen entsprechenden Antrag an den Prüfungsausschuss (PA) stellen.
Gleiches gilt auch, wenn du eine frühere Studienleistung (also einen Leistungsnachweis oder das Ergebnis einer Prüfung) anerkannt haben möchtest.
Vorher solltet du unbedingt mit deinen zwei studentischen Vertretern im Prüfungsausschuss oder mit dem FSR sprechen.
Die Vorsitzenden der Prüfungsausschüsse sind Prof. Baier (Informatik) und Prof. Groh (Medieninformatik), in dringlichen Fällen kannst du dich direkt an sie wenden.

Wo: Prüfungsamt, INF 3039/3040 \\
Wann: Di, Do: 12.30 - 15.00 Uhr \\
Mi: 9.00 - 11.00 Uhr \\
Telefon: (0351) 463-38378

Prüfungsausschuss-Vorsitzende:

\begin{multicols}{2}
Prof. Dr. Christel Baier \\
Büro: INF 3006 \\
Telefon: (0351) 463-38548 \\
E-Mail: christel.baier@tu-dresden.de

Prof. Dr. Rainer Groh \\
Büro: INF 2064 \\
Telefon: (0351) 463-39178 \\
E-Mail: rainer.groh@tu-dresden.de
\end{multicols}
