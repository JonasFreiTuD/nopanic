\chapter*{Frequently Asked Questions}
\addcontentsline{toc}{chapter}{Frequently Asked Questions \hspace*{.1cm} \keys{must read}}
\label{sec:faq}
\minisec{Wie schreibe ich mich in Prüfungen ein?}
In jExam \link{https://jexam.inf.tu-dresden.de} gibt es am Ende des Semesters eine Liste mit allen Prüfungen. Dort kannst du dich in die Prüfung einschreiben.

\minisec{Wie lange kann ich mich wieder austragen?}
Du kannst dich ohne ärztlichen Attest bis zu drei Tage vor der Prüfung über jExam austragen. Falls du erkrankt bist, kannst du mit einem Attest zum Prüfungsamt gehen. Dann wirst du auch aus deiner Prüfung ausgetragen.

\minisec{Wie baue ich mir meinen Stundenplan?}
Du suchst dir die Lehrveranstaltungen aus dem Katalog aus, die du besuchen möchtest. Dann suchst du dir Termine raus und versuchst kollisionen zu vermeiden.

\minisec{Was muss ich tun, um bafög zu bekommen?}
Das Studierendenwerk hat eine Anleitung und bietet auch entsprechende Beratung an.

\minisec{Wie und wann melde ich mich für Unisport an?}
Auf der Seite des Unisportzentrums gibt es eine Liste mit allen Angeboten und Terminen, an denen die Einschreibung für den Kurs beginnt. Dann kannst du auf der Seite das Formular schnell ausfüllen und bist dann eingeschrieben.

\minisec{Was sollen diese ganzen Portale?}
In guter Tradition gibt es für jeden Zweck mindestens 5 verschiedene Portale. Das muss so.

\minisec{Wie finde ich Freund\_Innen?}
In der ESE hast du die Chance mit vielen Menschen mit dem gleichen Problem in Kontakt zu kommen. Ansonsten gibt es viele Hochschulgruppen, bei denen du mitmachen kannst und Menschen kennen lernst.

\minisec{Was kann ich neben dem Studium machen?}
Du kannst die in Hochschulgruppen, im FSR, oder irgendwo ehrenamtlich engagieren. Alternativ kannst du natürlich Geld verdienen und eine der vielen freien Stellen besetzen. Über den Verteiler extern@ifsr.de werden viele Stellenangebote an Studierende verteilt. Falls dort nichts für dich dabei ist, schau mal bei der STAV (studentische Arbeitsvermitlung)~\link{https://www.stav-dresden.de} vorbei.

\minisec{Was muss ich während des Semesters machen?}
Prinzipell nichts außer dich am Ende Rückmelden, um weiter immatrikuliert zu sein

\minisec{Was sollte ich während des Semesters machen?}
Du solltest die Übungen und Vorlesungen besuchen, um am Ende die Prüfungen zu bestehen.

\minisec{Was muss ich mit meiner neuen Wohnung machen?}
Du musst einen Wohnsitz anmelden. Achtung: Die Stadt Dresden erhebt eine Zweitwohnsitzsteuer, es ist also sinnvoll, den Hauptwohnsitz nach Dresden zu verlegen.

\minisec{Wie funktioniert das mit dem WLAN?}
Das ZIH hat Anleitungen für viele Systeme zum Einrichten des Eduroam. Falls die Anleitungen dein Problem nicht lösen, kannst du zum ServiceDesk gehen.

\minisec{Wie lern ich Programmieren?}
Der ifsr bietet Programmierkurse für fast alle Sprachen an, die du im Studium benötigen wirst. \link{https://ifsr.de/kurse}

\minisec{Was ist eine Rekursion?}
\label{minisec:faq}
\ref{minisec:faq}

\minisec{Was ist AQUA?}
Abkürzung für Allgemeine Qualifikation. Ist ein Bestandteil deines Studiums. Genaueres: Siehe Modulübersicht in diesem Heft.

\minisec{Was für Bücher brauche ich?}
Eigentlich keine, aber falls du mal etwas genauer nachlesen möchtest oder etwa nicht ganz verstanden hast, gibt es zu allen Themen genug Bücher in der SLUB.

\minisec{Wo kann ich Dinge drucken?}
Es gibt Copyshops über die ganze Stadt verteilt, der FSR bietet aber auch einen Druckservice im FSR Büro.

\minisec{Wie komme ich an Altklausuren?}
Es gibt eine Sammlung auf dem FTP-Server des FSR~\link{https://ftp.ifsr.de/klausuren} (nur aus dem Uni-Netz oder VPN erreichbar), aber einige Lehrstühle stellen auch welche auf ihre Webseiten.
