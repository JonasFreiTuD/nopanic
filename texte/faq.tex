\newcommand{\fancypageref}[1] {%
    \changemenucolor{gray}{br}{named}{ese_bg_color}%
    \changemenucolor{gray}{txt}{named}{ese_bg_color}%
    \keys{Seite \pageref{#1}}%
    \changemenucolor{gray}{br}{named}{ese_bg_color}%
    \changemenucolor{gray}{txt}{named}{ese_bg_color}%
}

\chapter*{Frequently Asked Questions}
\addcontentsline{toc}{chapter}{Frequently Asked Questions \hspace*{.1cm} \keys{must read}}
\label{sec:faq}
\minisec{Was muss ich während des Semesters machen?}
Prinzipiell nichts außer dich am Ende zurückzumelden, um weiter immatrikuliert zu sein.

\minisec{Was sollte ich während des Semesters machen?}
Du solltest die Übungen und Vorlesungen besuchen, um am Ende die Prüfungen zu bestehen.

\minisec{Wie schreibe ich mich in Prüfungen ein?}
Für die schriftlichen Prüfungen in den Pflichtmodulen läuft die Anmeldung über jExam \link{https://jexam.inf.tu-dresden.de/}.
Dort gibt es am Ende der Vorlesungszeit eine Liste mit allen Prüfungen, in die du dich einschreiben kannst.

\minisec{Wie kann ich mich von Prüfungen abmelden?}
Du kannst dich ohne Angabe von Gründen bis drei Werktage von schriftlichen Prüfungen über jExam abmelden. Falls du erkrankt bist, kannst du auch noch danach von der Prüfung zurücktreten. Der Krankenschein ist dann dem Prüfungsamt zügig vorzulegen.

\minisec{Ich habe eine Prüfung nicht bestanden, wann muss ich sie wiederholen?}
Nicht-bestandene Modulprüfungen müssen innerhalb eines Jahres einmal wiederholt werden. Beim Nichtbestehen muss eine zweite Wiederholung zum nächstmöglichen Prüfungstermin abgelegt werden. Danach gilt die Modulprüfung endgültig als nicht bestanden. 
Durch COVID-19 kann es sein, dass es Sonderregulungen diesbezüglich gibt. Hier gilt es, sich stehts informiert zu halten z.B. auf der Corona-Informationsseite der TU Dresden \link{https://tu-dresden.de/tu-dresden/gesundheitsmanagement/information-regarding-covid-19-coronavirus-sars-cov-2}. 
Übrigens: Eine aus mehreren Prüfungsleistungen bestehende Modulprüfung kann bei entsprechender Gewichtung der Noten bestanden sein, auch wenn eine der Prüfungsleistungen nicht bestanden ist.

\minisec{Wie lange darf man überhaupt studieren?}
In Kurzform: Regelstudienzeit + 4 Semester. Es gibt aber einige Möglichkeiten, wie z.B. Urlaubs- und Gremiensemester, um diese Dauer zu verlängern.

\minisec{Wie baue ich mir meinen Stundenplan?}
Für den Anfang bekommst du fertige Stundenpläne von uns, aus denen du dann einfach einen Stundenplan am Tag der Einschreibung auswählen kannst. Ab dem zweiten Semester bieten die Studienablaufpläne eine Empfehlung, welche Module man in welchem Semester besuchen sollte. Du suchst dir die Lehrveranstaltungen aus dem Katalog aus, die du besuchen möchtest. Dann suchst du dir Termine heraus und versuchst, Kollisionen zu vermeiden.

\minisec{Was ist AQua?}
Diese Abkürzung steht für Allgemeine Qualifikation. Es ist ein Bestandteil deines Studiums.~\fancypageref{lec:aqua}

\minisec{Was sollen diese ganzen Portale?}
In guter Tradition gibt es für jeden Zweck mindestens 5 verschiedene Portale. Das muss so. Für dich sind eigentlich nur folgende wichtig:
\begin{itemize}
\item jExam für die Einschreibung in Lehrveranstaltungen und Prüfungen
\item Selma~\link{https://selma.tu-dresden.de} für das Herunterladen der Immatrikulationsbescheinigung
\item OPAL wird besonders in der Online-Lehre häufig für die Einschreibung und Organisation von Lehrveranstaltungen genutzt.
\end{itemize}

\minisec{Wie funktioniert das mit dem WLAN?}
Das ZIH hat Anleitungen~\link{https://tu-dresden.de/zih/dienste/service-katalog/arbeitsumgebung/zugang_datennetz/wlan-eduroam} für viele Systeme zum Einrichten des Eduroam. Falls die Anleitungen dein Problem nicht lösen, hilft dir der ServiceDesk weiter.

\minisec{Was muss ich tun, um BAföG zu bekommen?}
Das Studierendenwerk~\fancypageref{sec:stuwe} ist für die Bearbeitung der Anträge zuständig und bietet auch entsprechende Beratung an. Daneben kannst du dich mit deinen Fragen auch an den StuRa~\fancypageref{sec:stura} wenden.

\minisec{Wie und wann melde ich mich für Unisport an?}
Das funktioniert über das Universitätssportzentrum, das alle Kurse verwaltet.~\fancypageref{sec:sport} 
% Auf der Seite des Universitätssportzentrums gibt es eine Liste mit allen Angeboten und Terminen, an denen die Einschreibung beginnt. Dann kannst du auf der Seite das Formular schnell ausfüllen und bist eingeschrieben.

\minisec{Wie und wann melde ich mich für Sprachkurse an?}
Hierfür gibt es die Webseite des Lehrzentrum Sprachen und Kulturen -- kurz LSK.~\fancypageref{sec:sprache}

\minisec{Wie finde ich Freund\_innen?}
Während der ESE hast du die Chance, mit vielen Menschen in Kontakt zu kommen, die momentan wohl vor einem ähnlichen Problem stehen wie du.
Ansonsten gibt es immer mal wieder Veranstaltungen~\fancypageref{cha:veranstaltungen}, bei denen du Mitstudierende kennenlernen kannst.
Oder du schaust dich außerhalb deines Studiums mal um. Dein Leben besteht ja nicht nur aus Studieren. ;)

\minisec{Wer oder was ist der FSR?}
Der Fachschaftsrat, oder kurz FSR, vertritt dich und deine Interessen in der Uni. Nebenbei organisieren wir aber auch die ein oder andere Veranstaltung, zum Beispiel die ESE. Mehr über uns findest du auf~\fancypageref{sec:fachschaftsrat}.

\minisec{Was kann ich neben dem Studium machen?}
Du kannst dich in Hochschulgruppen~\fancypageref{sec:hsg}, der AG DSN~\link{https://www.agdsn.de}, im FSR, oder irgendwo ehrenamtlich engagieren. Alternativ kannst du natürlich Geld verdienen und eine der vielen freien Stellen besetzen. Über den Verteiler \texttt{extern@ifsr.de}~\link{https://lists.ifsr.de/listinfo/extern} werden viele Stellenangebote an Studierende verteilt. Falls dort nichts für dich dabei ist, schau mal bei der STAV (studentische Arbeitsvermittlung)~\link{https://www.stav-dresden.de} vorbei.

\minisec{Ist das Semesterticket das ultimative Machtwerkzeug?}
Leider nein. Es gibt eine handvoll Einschränkungen (Fahrräder darfst du z.B. nur zu bestimmten Zeiten gratis mitnehmen), die dir aber auf den Seiten des StuRa~\link{https://www.stura.tu-dresden.de/semesterticket} erklärt werden.

\minisec{Was muss ich mit meiner neuen Wohnung machen?}
Du solltest vermutlich deine Miete regelmäßig zahlen. Ansonsten vergiss nicht, dich bei der Stadt Dresden umzumelden! \fancypageref{sec:ummelden}
% Du musst einen Wohnsitz anmelden. Achtung: Die Stadt Dresden erhebt eine Zweitwohnsitzsteuer, es ist also sinnvoll, den Hauptwohnsitz nach Dresden zu verlegen. Zugezogene können außerdem von einer Umzugsbeihilfe profitieren.

\minisec{Wo lerne ich Programmieren?}
In den Vorlesungen leider gar nicht. Dort bekommst du in der Regel nur einen knappen Crash-Kurs, wie zum Beispiel in Algorithmen und Datenstrukturen für die Sprache C~\fancypageref{sec:aud}.
Der FSR bietet allerdings Programmierkurse für fast alle Sprachen an, die du im Studium benötigen wirst. \link{https://www.ifsr.de/kurse} Außerdem kannst du auch über LinkedIn Leaning \link{https://www.slub-dresden.de}, dass von der SLUB gereitgestellt wird, Programmierkurse machen.

\minisec{Wann lerne ich Game Development?}
Im Studium tendenziell gar nicht. Dafür gibt es keine wirklichen Veranstaltungen abgesehen von ein paar Komplexpraktika im späteren Studienverlauf. Die Grundlagen von Grafikrendering lernst du in \enquote{Einführung in die Computergrafik}~\fancypageref{lec:ecg}, aber das war's auch schon. :(

\minisec{Was ist eine Rekursion?}
\label{minisec:faq}
Wenn du mehr darüber wissen willst, schau mal auf~\fancypageref{minisec:faq}.

\minisec{Was für Bücher brauche ich?}
Eigentlich keine, aber falls du mal etwas genauer nachlesen möchtest oder etwas nicht ganz verstanden hast, gibt es zu allen Themen genug Bücher und eBooks in der SLUB.~\fancypageref{sec:slub}

\minisec{Wo kann ich Dinge drucken?}
Es gibt viele Orte an denen gedruckt werden kann. Zum einen bietet bspw. die SLUB einen Druckservice an. Zum anderen gibt es aber auch auf dem Campus und in der ganzen Stadt Copyshops.

\minisec{Wie komme ich an Altklausuren?}
Es gibt eine Sammlung auf dem FTP-Server des FSR~\link{https://ftp.ifsr.de/klausuren} (nur aus dem Uni-Netz oder VPN erreichbar), aber einige Lehrstühle stellen auch welche auf ihre Webseiten.

\minisec{Was ist denn diese s-Nummer?}
Mit der s-Nummer ist dein ZIH-Login gemeint. Dieses System wurde mittlerweile umgestellt und dein ZIH-Login besteht nun aus Buchstaben und Zahlen, so wie auf der Webseite des ZIH beschrieben~\link{https://tu-dresden.de/zih/dienste/service-katalog/zugangsvoraussetzung}. Vorher bestand der Login nur aus einer 7-stelligen Nummer mit einem \enquote{s} voran -- daher \enquote{s-Nummer}.

\minisec{Wofür stehen überall diese gelben Fahrräder herum?}
Die gelben Fahrräder gehören zum Fahrradverleihsystem MOBIbike, das ein Angebot der Dresdner Verkehrsbetriebe AG (DVB) und nextbike ist. Ihr könnt die MOBIbikes und sogar nextbikes in den meisten anderen deutschen nextbike-Städten zu einer besonderen Kondition ausleihen. Wie das genau funktioniert und weitere Details findet ihr auf den Webseiten des StuRa~\link{https://www.stura.tu-dresden.de/nextbike}.

\minisec{Warum befindet sich im Teich hinter dem APB kein Wasser?}
Auf die freie Fläche hinter dem Andreas-Pfitzmann-Bau soll ein neues Gebäude für das Deutsche Zentrum für Luft- und Raumfahrt (DLR) gebaut werden. Für diesen Bau musste der Teich entwässert werden.~\fancypageref{sec:apb}
