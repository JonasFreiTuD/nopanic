\addchap{Studienalltag}

Mit dem Studium beginnt ein neuer Lebensabschnitt. Es kommen neue Aufgaben und Herausforderungen auf dich zu, die es zu meistern gilt. Auf den folgenden Seiten möchten
wir dir einen Eindruck darüber verschaffen, wie das Studium aufgebaut ist und wie das Lernen an der Uni funktioniert. Lies dir bitte unbedingt auch mal deine Studienordnung
und die Prüfungsordnung durch.

\minisec{Module}
Im Verlauf deines Studiums musst du zahlreiche sogenannte Module erfolgreich absolvieren. Ein Modul kann mehrere Lehrveranstaltungen beinhalten. Das können Vorlesungen,
Übungen, Praktika oder auch Seminare sein. Viele Module bestehen nur aus einer Vorlesung und einer dazugehörigen Übung. Du schließt ein Modul ab, indem du die Modulprüfung
bestehst. Eine Modulprüfung kann sich aus einer oder mehreren Prüfungsleistungen (z.B. Klausur) zusammensetzen. Manchmal muss zunächst eine Prüfungsvorleistung erbracht werden, um
überhaupt an einer Prüfung teilnehmen zu dürfen. Für die einzelnen Module ist in der Anlage 2 zur Studienordnung (Modulbeschreibungen) genau geregelt, welche Prüfungsleistungen zu
erbringen sind.
Jedes Modul hat eine ausgeschriebene Anzahl an Leistungspunkten (LP). Dabei entspricht ein LP einer Arbeitsbelastung von 30 Stunden. Wenn ein Modul 5 LP bringt, heißt das also,
dass über das Semester verteilt 150 Stunden Arbeit anstehen. Diese Arbeitsbelastung setzt sich zusammen aus Präsenzzeit (Zeit, die du tatsächlich in Vorlesungen/Übungen an der Uni verbringst),
Zeit zur Vor- und Nachbereitung der Lehrveranstaltungen (Selbststudium), Prüfungsvorbereitung und der Prüfung selbst. Die Leistungspunkte für ein Modul werden erst dann anerkannt, wenn
die Modulprüfung bestanden ist.

\minisec{Stundenplan}

An der Uni gibt es ein so genanntes Lehrangebot, das kurz vor Beginn jedes Semesters veröffentlicht wird.
Du findest diese bereits nach Semestern sortierte Liste von Lehrveranstaltungen online auf der Seite der Fakultät \link{https://inf.tu-dresden.de/}.
Ab dem zweiten Semester besteht deine Aufgabe darin, dir aus diesem Angebot deinen Stundenplan zu basteln.
Für den Anfang bekommst du jedoch zum Eingewöhnen fertige Stundenpläne von uns, aus denen du dann einfach einen zur Einschreibung in jExam auswählen kannst.

Während Vorlesungen generell einen festen Termin haben, kannst du dich ab dem zweiten Semester flexibel in die Übungen eintragen.
Schreib dich bei jExam \link{https://jexam.inf.tu-dresden.de/} einfach für die Übungsstunden deiner Wahl ein.
Stellst du später jedoch fest, dass dein Übungsleiter die Qualitäten einer Schlaftablette aufweist oder dir die Übung zu voll ist, zögere nicht die Übung zu wechseln.

Wenn du dir das Lehrangebot anschaust, wirst du auf die Abkürzung SWS stoßen. SWS steht für Semesterwochenstunden und gibt den Zeitaufwand für eine Lehrveranstaltung an.
SWS treffen dabei lediglich eine Aussage über die Präsenzzeit an der Uni. Die Zeit zur Vor- und Nachbereitung wird dabei nicht berücksichtigt.
Die Angabe 1 SWS bedeutet, dass die Lehrveranstaltung während der Vorlesungszeit wöchentlich durchschnittlich 45 min lang gelehrt wird. Eine Lehrveranstaltung mit 4 SWS wird entsprechend
pro Woche 3 Stunden gelehrt. Eine Lehreinheit an der Uni dauert 90 Minuten und wird Doppelstunde (DS) genannt. Eine Veranstaltung mit 4 SWS findet also 2-mal wöchentlich statt.
Etwas komplizierter ist es, wenn eine Lehrveranstaltung tatsächlich nur 1 SWS umfasst. Dann findet die Veranstaltung nur 14-tägig statt und man muss genau schauen, ob die Veranstaltung jeweils in
geraden oder ungeraden Kalenderwochen stattfindet. Im Stundenplan wirst du dann die Bezeichnungen "1. Woche" oder "2. Woche" finden. Diese haben nichts mit den Wochen seit Semesterbeginn zu tun!
"1. Woche" bedeutet, dass die Lehrveranstaltung in jeder ungeraden Kalenderwoche stattfindet und "2. Woche" steht analog für gerade Kalenderwochen.

\minisec{Vorlesung}

In diesen Veranstaltungen erlebst du meistens Professoren live.
Die Zahl der Zuhörer ist in der Regel zehn Mal so groß wie die Anzahl der Schüler in einer Unterrichtsstunde. Dadurch kann natürlich nicht auf jeden Studenten eingegangen werden.
Sollte dir jedoch etwas unklar sein, stelle ruhig Fragen. Oft versteht der Großteil der anderen auch nichts.
Außerdem ist es ratsam, dem Stoff stets zu folgen, da die vermittelte Stoffmenge oftmals gewaltig ist. Sich darüber zu beschweren ist sinnlos, da der Lehrplan für die Professoren mehr oder minder vorgeschrieben ist.
Gerade deshalb hat man nur 20 Wochenstunden, da für die Nachbereitung einer Vorlesung mindestens die gleiche Zeit veranschlagt werden sollte.
Allerdings beschweren solltest du dich über schlechte Tafelbilder, undeutliche und leise Aussprache, sowie mangelnde Vorbereitung der Vorlesung.
Professoren sind meist nicht Professoren, weil sie gute Didaktiker sind, sondern weil sie gut forschen können.
Welche Vorlesung du in welchem Semester besuchen solltest, findest du im jeweiligen Studienablaufplan deines Studiengangs
(Bachelor Informatik \link{https://www.verw.tu-dresden.de/AmtBek/PDF-Dateien/2016-06/11soBA24.04.2016.pdf}, Bachelor Medieninformatik \link{https://www.verw.tu-dresden.de/AmtBek/PDF-Dateien/2016-06/11soBAMI24.04.2016.pdf}, Diplom Informatik \link{https://tu-dresden.de/die_tu_dresden/fakultaeten/fakultaet_informatik/studium/dateien/studien_und_pruefungsordnungen/dipl_inf_so_app1_de.pdf}) oder im Vorlesungsverzeichnis auf der Seite der Fakultät \link{https://tu-dresden.de/ing/informatik/studium/lehre}.


\minisec{Übung}

Übungen werden zu fast allen Vorlesungen angeboten und dienen dazu, Aufgaben zum aktuellen Vorlesungsstoff zu bearbeiten. Klausuren orientieren sich häufig an den Übungsaufgaben, deshalb solltest du die Übungen
regelmäßig besuchen. Die Übungen werden meistens von Studenten aus höheren Semestern oder auch Lehrstuhlmitarbeitern gehalten, nicht vom Professor.
Das hat nämlich auch den Vorteil, dass man ja bekanntlich viele Dinge besser versteht, wenn man sie noch einmal aus einem anderen Mund erklärt bekommt.
Die jeweils aktuellen Übungsaufgaben findest du auf der Seite des jeweiligen Dozenten, oft unter den Stichworten Teaching oder Lehre.
Es wird erwartet, dass du dir die Aufgaben bereits vor der Übung anschaust, um dann Lösungsansätze zu diskutieren und Fragen stellen zu können.


\minisec{Praktikum}

Das erste Praktikum erwartet dich bereits in der vorlesungsfreien Zeit des ersten Semesters – plane deinen Urlaub also lieber nicht zu schnell!
Dort wirst du im Einführungspraktikum – Robolab dein Können unter Beweis stellen. Diplomer müssen zusätzlich noch das Strategiespielpraktikum absolvieren.
Ein ganzes Praktikumssemester ist nur für Diplomstudenten im 7. Semester Pflicht.
Natürlich ist es trotzdem empfehlenswert, Praktika bei echten Firmen außerhalb der Fakultät in den Semesterferien zu machen, das steigert nicht nur deine Jobchancen,
sondern zeigt dir auch, ob deine Studienwahl tatsächlich die Richtige war.


\minisec{Prüfungen}

Direkt an die Vorlesungszeit schließt die Prüfungszeit an – die wohl stressigste Zeit im Leben eines Studenten.
Die genauen Prüfungstermine findest du für das Wintersemester meist etwa Anfang Januar auf der Homepage der Fakultät \link{https://tu-dresden.de/ing/informatik/studium/news} oder direkt beim Prüfungsamt \link{https://tu-dresden.de/ing/informatik/studium/pruefungsorganisation}.
Im Laufe des Semesters hast du die Gelegenheit, dich dafür (innerhalb der Einschreibefrist) über jExam einzuschreiben.
Dort hast du auch die Möglichkeit, dich bis zu drei \emph{Werk}tage vor der Prüfung wieder auszutragen. Dann kannst du die Prüfung auch in einem späteren Semester schreiben, was aber natürlich
nicht zum Regelfall werden sollte. Für mündliche und sonstige Prüfungen gilt eine Frist von 14 Tagen.
Solltest du aufgrund eines Rücktritts innerhalb der Frist oder einer plötzlichen Erkrankung von der Prüfung ausscheiden, kannst du dich auf der Seite des Prüfungsamtes informieren,
welche Nachweise (Atteste) du im Prüfungsamt innerhalb welcher Frist einreichen musst \link{https://tu-dresden.de/ing/informatik/studium/pruefungsorganisation/pruefungen/abmelden-ruecktritt-krankheit}.
Prüfungen werden mit Noten bewertet, alles außer \textbf{5} ist bestanden und bestandene Prüfungen können nicht wiederholt werden.
Bist du durchgefallen, hast du die Möglichkeit, die Prüfung innerhalb von zwei Semestern zu wiederholen. Erst wenn du auch die zweite Wiederholungsklausur versemmelt hast, wirst du exmatrikuliert.
Genauere Informationen zu dieser Thematik findest du stets in der Prüfungs- bzw. der Studienordnung, die du dir unbedingt mal angeschaut haben solltest.
Eine erste Matheprüfung erwartet dich übrigens bereits im Dezember.


\minisec{Leistungsnachweis}

Bei manchen Prüfungen erhältst du neben der Note einen Leistungsnachweis (oder kurz: Schein).
Dazu zählen unter anderem die Sprachkurse, die Forschungslinie und z.T. Nebenfachprüfungen. Diese Scheine brauchst du, um dir diese Leistungen im Prüfungsamt anrechnen lassen zu können.


\minisec{Sprachausbildung}\label{sec:sprachausbildung}

Es werden an der TU Dresden Kurse für fast alle möglichen (und unmöglichen) Sprachen angeboten.
Zu diesem Zweck gibt es zwei Zentren für die Sprachausbildung: Das \glqq Lehrzentrum Sprachen und Kulturen\grqq\ (LSK) und \glqq TUD Institute of Advanced Studies\grqq\ (TUDIAS).
Das Sprachangebot der beiden Einrichtungen ähnelt sich sehr stark.
Du hast für diverse Sprachkurse ein Budget an Semesterwochenstunden (insgesamt 10 SWS), die du ausgeben kannst, wie du willst.
Für dein Studium zum Bachelor der (Medien-)Informatik sind Sprachkurse generell optional, aber auf jeden Fall empfehlenswert.
Für Diplomstudenten sind 2 Semester Englisch im Laufe des Studiums Pflicht.
Studierst du allerdings Bachelor Informatik und möchtest danach mit dem Master Informatik an der TU Dresden weitermachen, wirst du für den Master das Sprachniveau B2 in Englisch nachweisen müssen,
also kann es sich auch für dich anbieten, die entsprechenden Sprachkurse zu besuchen.
Die Einschreibung für einen Sprachkurs erfolgt online \link{http://sprachausbildung.tu-dresden.de} mit deinem ZIH-Login.
Sobald die Kurse freigeschaltet sind, solltest du dich jedoch stark beeilen, denn die beliebten Kurse sind meist innerhalb weniger Minuten voll.
Weitere Infos findest du unter \link{https://tu-dresden.de/lsk/lskonline} und \link{http://www.tudias.de/de/Sprachschule.html}.
