\addchap{Der Studienbetrieb}

\minisec{Die Grundbegriffe des Studiums in kurzen Worten erklärt.}

Wer \glqq frisch\grqq\ aus der Schule kommt, kennt als Lehrform vor allem den Dialog.
Üblicherweise versucht der Lehrer in der Schule, auf die Denkweise und das Arbeitstempo der Schüler einzugehen, unterhält sich mehr mit ihnen, als dass er ihnen einen Vortrag hält.
Am Ende der Stunde hat zumindest ein großer Teil der Schüler den Stoff verstanden.
An der Uni gibt es diese Lehrmethode nicht - dafür aber einige andere, an die man sich auch gewöhnen kann.
Hier wird viel Wert auf Eigenständigkeit gelegt, ein \glqq an die Hand genommen werden\grqq\ wie in der Schule, gibt es nicht mehr.
Das ist nicht die einzige Neuerung, die im Studienalltag auf euch zukommt.

\minisec{Der Stundenplan}

Eigentlich fangen die Veränderungen schon beim Stundenplan an.
Es gibt ein so genanntes Lehrangebot, das kurz vor Beginn jedes Semesters veröffentlicht wird.
Ihr findet diese Liste von Lehrveranstaltungen online auf der Seite der Fakultät \link{http://inf.tu-dresden.de/}.
Glücklicherweise wird an dieser Stelle schon nach den entsprechenden Semestern sortiert.
Eure Aufgabe besteht nun darin, aus dem Angebot einen Stundenplan zu basteln.
Im ersten Semester bekommt ihr jedoch zum Eingewöhnen fertige Stundenpläne von uns, aus denen ihr dann einfach zur Einschreibung auswählen könnt.
Ab dem zweiten Semester liegt diese Aufgabe bei euch.
Für Vorlesungen gibt es generell jeweils nur einen Termin, den müsst ihr so einplanen, wie er ist.
Bei den Übungen ist das ein ganzes Stück flexibler.
Ihr schreibt euch bei jExam \link{http://jexam.inf.tu-dresden.de/} für eine von den für ein Fach angebotenen Übungsstunden ein.
Ihr seit nicht gezwungen, in eurer Übung zu bleiben.
Sollten dort zu viele Leute sitzen (mehr als 30 sind schon eher unpraktisch) oder sollte der Übungsleiter die Qualitäten einer Schlaftablette aufweisen, scheut euch nicht in eine andere Übung zu wechseln.

\minisec{Die Vorlesung}

In diesen Veranstaltungen erlebt ihr meistens Professoren live.
Die Zahl der Zuhörer ist in der Regel zehn Mal so groß wie die Anzahl der Schüler in einer Unterrichtsstunde.
Das schränkt die Dialogmöglichkeiten unheimlich ein.
Es ist kaum machbar, dass jeder seine Fragen in der Vorlesung beantwortet bekommt.
Traut euch aber trotzdem, Fragen zu stellen.
Geht davon aus, dass mindestens 50\% der anderen Hörer auch nichts verstehen und sich nur nicht trauen, die Frage zu stellen.
Die in einem Semester zu bewältigende Stoffmenge ist gewaltig im Vergleich zu dem Stoff, der in der Schule durch genommen wird.
Sich über die Geschwindigkeit des Vorgehens aufzuregen ist sinnlos; auch die Lehrpläne der Professoren sind mehr oder minder fest vorgeschrieben.
Aber da man sich im Studium auf einige wenige Fächer konzentriert und nur ca. 20 bis 25 Wochenstunden zu besuchen hat, kommt man schon zurecht.
Auch hat man deshalb nur 20 Wochenstunden, da man für die Nachbereitung einer Vorlesung mindestens die gleiche Zeit veranschlagen sollte.
Beschweren allerdings könnt und solltet ihr euch aber durchaus über unleserliche und wirre Tafelbilder, zu schnelles Anschreiben an die Tafel, undeutliche und leise Aussprache und mangelhafte Vorbereitung der Vorlesung (äußert sich in schlechter Beweisführung und unverständlichen Antworten auf Zwischenfragen).
Professoren sind meist nicht Professoren, weil sie gute Didaktiker sind, sondern weil sie gut forschen können.
Das bedeutet dann eben auch, dass ein durchschnittlicher Gymnasiallehrer in Sachen Wissensvermittlung in der Regel besser ist als durchschnittlicher Hochschulprofessor.
Welche Vorlesung ihr in welchem Semester besuchen solltet, findet ihr im jeweiligen Studienablaufplan eures Studiengangs (Bachelor Informatik \link{https://www.inf.tu-dresden.de/content/study/regulations/download/ba-inf/2009/study.app.2.de.pdf}, Bachelor Medieninformatik \link{https://www.inf.tu-dresden.de/content/study/regulations/download/ba-minf/2009/study.app.2.de.pdf}, Diplom Informatik\link{http://www.inf.tu-dresden.de/content/study/regulations/download/inf/2010/study.de.pdf}) oder im Vorlesungsverzeichnis auf der Seite der Fakultät \link{http://www.inf.tu-dresden.de/index.php?node_id=2709&ln=de}.

\minisec{Die Übungen}

Zu fast allen Vorlesungen werden auch entsprechende Übungen angeboten.
Dort werden Aufgaben zum aktuellen Vorlesungsstoff bearbeitet.
Es wird davon ausgegangen, dass sich die Studenten schon im Voraus mit diesen Aufgaben beschäftigt haben und eigene Lösungsvorschläge diskutieren können.
Oft könnt ihr brennende Fragen auch im Anschluss an eine Übung in Ruhe mit dem Übungsleiter besprechen.
Selten haben die Dozenten und Professoren selbst die Zeit, eine solche Übung durchzuführen, so dass dies meist andere Mitarbeiter übernehmen.
Das hat den Vorteil, dass man ja bekanntlich viele Dinger besser versteht, wenn man sie noch einmal aus einem anderen Mund erklärt bekommt.
Häufig orientieren sich die Klausuraufgaben an den Übungen, allein schon deshalb lohnt es sich, regelmäßig zur Übung zu gehen.
Die Aufgaben findet ihr auf der Seite des jeweiligen Dozenten, oft unter Stichworten wie Teaching oder Lehre.

\minisec{Das Praktikum}

Hier soll nun der Beweis geführt werden, dass ihr mit dem in den Veranstaltungen vermitteltem Wissen außer Vergessen auch noch etwas anderes anfangen könnt.
Bereits in den Semesterferien des ersten Semesters (plant euren Urlaub daher nicht allzu schnell) seid ihr beim Einführungspraktikum - Robolab für Bachelor-, Strategiespielpraktikum für Diplomstudenten - gefordert.
Ein Praktikum außerhalb der Uni ist nicht obligatorisch (außer für Diplomstudenten im 7. Semester).
Es versteht sich aber von selbst, dass ihr davon in den Semesterferien später Gebrauch machen solltet.
Nicht zuletzt steigert ihr damit eure Chancen bei der späteren Jobsuche und für die meisten ist es eine willkommene Abwechslung.
Außerdem merkt ihr so am besten, ob ihr mit der (Medien-)Informatik das Richtige für euch gefunden habt, wofür ihr eigentlich studiert und worauf ihr euch noch besser konzentrieren solltet.

\minisec{Prüfungen}

Das vielleicht Schwierigste und Wichtigste zugleich im Leben eines Studenten sind die Prüfungen.
Sie werden normalerweise in der Prüfungszeit im Anschluss an die Vorlesungszeit geschrieben.
Genauere Informationen findet ihr zu dieser Thematik stets in der Prüfungs- bzw. der Studienordnung, die ihr euch unbedingt mal angeschaut haben solltet.
Wenn ihr eine Prüfung schreiben wollt, müsst ihr euch rechtzeitig für diese auch einschreiben.
Die Gelegenheit dazu habt ihr im Laufe des Semesters.
Die genauen Prüfungstermine findet ihr für das Wintersemester meist etwa Anfang Januar auf der Homepage der Fakultät \link{http://inf.tu-dresden.de/} unter \glqq Aktuelles\grqq\ oder direkt beim Prüfungsamt \link{http://www.inf.tu-dresden.de/index.php?node_id=904&ln=de}.
Die Einschreibung läuft auch hier über jExam.
In späteren Semestern existieren übrigens auch mündliche Prüfungen.
Prüfungen werden natürlich mit Noten bewertet, alles außer \textbf{5} ist bestanden und bestandene Prüfungen können nicht wiederholt werden.
Durchzufallen ist kein Beinbruch, ihr könnt euch problemlos ein Semester später für die Wiederholungsprüfung einschreiben (und sogar auch für eine zweite).
Erst wenn man auch die zweite Wiederholungsprüfung versemmelt wird man exmatrikuliert.
Bis zu drei Werk(!)tage vor einer Prüfung habt ihr allerdings auch die Möglichkeit, euch wieder auszutragen (bzw. gar nicht erst einzuschreiben) und könnt die Prüfung damit schieben und in einem späteren Semester schreiben.
Damit sollte man bestenfalls jedoch nicht direkt im ersten Semester anfangen.
Im Falle eines Rücktritts innerhalb der Frist oder einer plötzlichen Erkrankung könnt ihr euch auf der Seite des Prüfungsamtes informieren, welche Nachweise (Atteste) ihr im Prüfungsamt innerhalb welcher Frist einreichen müsst \link{http://www.inf.tu-dresden.de/index.php?node_id=906\&ln=de}.
An dieser Stelle ist auch anzumerken, dass eine erste Matheprüfung bereits gegen Mitte Dezember auf euch wartet, genauere Infos hierzu werden euch aber natürlich nocht mitgeteilt.

\minisec{Leistungsnachweise}
Um zu manchen Prüfungen überhaupt erst zugelassen zu werden, benötigt ihr sogenannte Leistungsnachweise bzw. Scheine (siehe Prüfungsordnung).
Ihr erhaltet einen Schein bei einem Praktikum oder bei Scheinklausuren.
Einschreibungen dazu erfolgen ebenfalls online über jExam.
Scheine unterscheiden sich von Prüfungen insofern, dass ihr unendlich oft versuchen könnt, einen Schein in einem Fach zu erhalten.
Aber Vorsicht: Scheine sind oft Voraussetzungen für Prüfungen und diese müssen bis zu einem bestimmten Zeitpunkt abgelegt sein.
In den meisten Fällen bestehen Vorleistungen allerdings aus der Abgabe einer bestimmten Anzahl an Übungsaufgaben.

\minisec{Sprachausbildung}\label{sec:sprachausbildung}

Es werden von der TUD Kurse für fast alle möglichen (und unmöglichen) Sprachen angeboten.
Zu diesem Zweck gibt es zwei Zentren für die Sprachausbildung: "Lehrzentrum Sprachen und Kulturen" (LSK) und "TUD Institute of Advanced Studies" (TUDIAS).
Das Sprachangebot der beiden Einrichtungen ähnelt sich sehr stark. Allerdings ist die Sprachausbildung am TUDIAS im Gegensatz zum LSK kostenpflichtig.
Ihr habt für diverse Sprachkurse ein Budget an Semesterwochenstunden (insgesamt 10 SWS), die ihr wie ihr wollt ausgeben könnt.
Für euer Studium zum Bachelor der (Medien-)Informatik sind Sprachkurse generell optional, aber auf jeden Fall empfehlenswert.
Für Diplomstudenten sind 4 SWS Englisch (also 2 Semester) Pflicht.
Die Einschreibung für einen Sprachkurs erfolgt online \link{https://sprachausbildung.tu-dresden.de} mit eurem ZIH-Login.
Sobald die Kurse freigeschaltet sind, solltet ihr euch jedoch stark beeilen, die beliebten Kurse sind meist innerhalb weniger Minuten voll.
Weitere Infos findet ihr unter \link{NONE}, \link{http://tu-dresden.de/die_tu_dresden/zentrale_einrichtungen/lsk/lskonline} und \link{http://www.tudias.de/de/Sprachschule.html}.
