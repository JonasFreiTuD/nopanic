\addchap{Studienbetrieb}

\minisec{Die Grundbegriffe des Studiums in kurzen Worten erklärt.}

Wer \glqq frisch\grqq\ aus der Schule kommt, kennt als Lehrform vor allem den Dialog.
Üblicherweise versucht der Lehrer in der Schule, auf die Denkweise und das Arbeitstempo der Schüler einzugehen, unterhält sich mehr mit ihnen, als dass er ihnen einen Vortrag hält.
Am Ende der Stunde hat zumindest ein großer Teil der Schüler den Stoff verstanden.
An der Uni gibt es diese Lehrmethode nicht - dafür aber einige andere, an die man sich auch gewöhnen kann.
Hier wird viel Wert auf Eigenständigkeit gelegt, ein \glqq an die Hand genommen werden\grqq\ wie in der Schule, gibt es nicht mehr.
Das ist nicht die einzige Neuerung, die im Studienalltag auf dich zukommt.

\minisec{Stundenplan}

An der Uni gibt ein so genanntes Lehrangebot, das kurz vor Beginn jedes Semesters veröffentlicht wird.
Du findest diese bereits nach Semestern sortierte Liste von Lehrveranstaltungen online auf der Seite der Fakultät \link{https://inf.tu-dresden.de/}.
Ab dem zweiten Semester besteht deine Aufgabe darin, aus dem Angebot einen Stundenplan zu basteln.
Für den Anfang bekommst du jedoch zum Eingewöhnen fertige Stundenpläne von uns, aus denen du dann einfach einen zur Einschreibung auswählen kannst.
Während Vorlesungen generell einen festen Termin haben, kannst du dich in die Übungen flexibel eintragen.
Schreib dich bei jExam \link{https://jexam.inf.tu-dresden.de/} einfach für die Übungsstunden deiner Wahl ein.
Stellst du später jedoch fest, dass dein Übungsleiter die Qualitäten einer Schlaftablette aufweist oder dir die Übung zu voll ist, zögere nicht die Übung zu wechseln.


\minisec{Vorlesung}

In diesen Veranstaltungen erlebst du meistens Professoren live.
Die Zahl der Zuhörer ist in der Regel zehn Mal so groß wie die Anzahl der Schüler in einer Unterrichtsstunde. Dadurch kann natürlich nicht auf jeden Studenten eingegangen werden. Sollte dir jedoch etwas unklar sein, stelle ruhig Fragen. Oft versteht der Großteil der anderen auch nichts.
Außerdem ist es ratsam dem Stoff stets zu folgen, da die Stoffmenge oftmals gewaltig erscheint. Sich darüber zu beschweren ist sinnlos, da der Lehrplan für die Professoren mehr oder minder vorgeschrieben ist.
Gerade deshalb hat man nur 20 Wochenstunden, da für die Nachbereitung einer Vorlesung mindestens die gleiche Zeit veranschlagt werden sollte.
Allerdings beschweren solltest du dich über schlechte Tafelbilder, undeutliche und leise Aussprache, sowie mangelnde Vorbereitung der Vorlesung. 
Professoren sind meist nicht Professoren, weil sie gute Didaktiker sind, sondern weil sie gut forschen können.
Welche Vorlesung du in welchem Semester besuchen solltest, findest du im jeweiligen Studienablaufplan deines Studiengangs (Bachelor Informatik \link{https://www.inf.tu-dresden.de/content/study/regulations/download/ba-inf/2009/study.app.2.de.pdf}, Bachelor Medieninformatik \link{https://www.inf.tu-dresden.de/content/study/regulations/download/ba-minf/2009/study.app.2.de.pdf}, Diplom Informatik \link{https://www.inf.tu-dresden.de/content/study/regulations/download/inf/2010/study.de.pdf}) oder im Vorlesungsverzeichnis auf der Seite der Fakultät \link{https://www.inf.tu-dresden.de/index.php?node_id=2709}.


\minisec{Übungen}

Übungen werden zu fast allen Vorlesungen angeboten und dienen dazu, Aufgaben zum aktuellen Vorlesungsstoff zu bearbeiten. Klausuren orientieren sich häufig an den Übungsaufgaben, deshalb solltest du die Übungen, die meist nicht vom Dozenten selbst gehalten werden, regelmäßig besuchen.
Das hat nämlich auch den Vorteil, dass man ja bekanntlich viele Dinge besser versteht, wenn man sie noch einmal aus einem anderen Mund erklärt bekommt.
Die jeweils aktuellen Übungsaufgaben findest du auf der Seite des jeweiligen Dozenten, oft unter den Stichworten Teaching oder Lehre.
Es wird erwartet dass du dir die Aufgaben bereits vor der Übung anschaust, um dann Lösungsansätze zu diskutieren und Fragen stellen zu können.


\minisec{Praktikum}

Das erste Praktikum erwartet dich bereits in den Semesterferien des ersten Semesters – plane deinen Urlaub also lieber nicht zu schnell!
Dort wirst du im Einführungspraktikum – Robolab, sowie die Diplomer zusätzlich im Strategiespielpraktikum, dein Können unter Beweis stellen.
Ein ganzes Praktikumssemester ist nur für Diplomstudenten im 7. Semester Pflicht.
Natürlich ist es trotzdem empfehlenswert Praktika bei echten Firmen außerhalb der Fakultät in den Semesterferien zu machen, das steigert nicht nur deine Jobchancen, sondern zeigt dir auch, ob eure Studienwahl tatsächlich die Richtige war.


\minisec{Prüfungen}

Direkt an die Vorlesungszeit schließt die Prüfungszeit an – das wohl Schwierigste im Leben eines Studenten.
Die genauen Prüfungstermine findest du für das Wintersemester meist etwa Anfang Januar auf der Homepage der Fakultät \link{https://inf.tu-dresden.de/} unter \glqq Aktuelles\grqq\ oder direkt beim Prüfungsamt \link{https://www.inf.tu-dresden.de/index.php?node_id=904}.
Im Laufe des Semesters hast du die Gelegenheit dich dafür rechtzeitig einzuschreiben, dies erfolgt auch hier über jExam.
Dort hast du auch die Möglichkeit, dich bis zu drei \emph{Werk}tage vor der Prüfung wieder auszutragen. Dann kannst du die Prüfung auch in einem späteren Semester schreiben, was aber natürlich nicht zum Regelfall werden sollte.
Solltest du aufgrund eines Rücktritts innerhalb der Frist oder einer plötzlichen Erkrankung von der Prüfung ausscheiden, kannst du dich auf der Seite des Prüfungsamtes informieren, welche Nachweise (Atteste) du im Prüfungsamt innerhalb welcher Frist einreichen musst \link{https://www.inf.tu-dresden.de/index.php?node_id=906}.
Prüfungen werden mit Noten bewertet, alles außer \textbf{5} ist bestanden und bestandene Prüfungen können nicht wiederholt werden.
Bist du durchgefallen, hast du die Möglichkeit die Prüfung innerhalb von zwei Semestern zu wiederholen. Erst wenn du auch die zweite Wiederholungsklausur versemmelt hast, wirst du exmatrikuliert.
Genauere Informationen findest du zu dieser Thematik stets in der Prüfungs- bzw. der Studienordnung, die du dir unbedingt mal angeschaut haben solltet.
Eine erste Mathe-Prüfung erwartet dich übrigens bereits im Dezember.


\minisec{Leistungsnachweise}
Um zu manchen Prüfungen überhaupt erst zugelassen zu werden, benötigst du sogenannte Leistungsnachweise bzw. Scheine (siehe Prüfungsordnung).
Du erhältst einen Schein bei einem Praktikum oder bei Scheinklausuren.
Einschreibungen dazu erfolgen ebenfalls online über jExam.
Scheine unterscheiden sich von Prüfungen insofern, dass du unendlich oft versuchen kannst, einen Schein in einem Fach zu erhalten.
Aber Vorsicht: Scheine sind oft Voraussetzungen für Prüfungen und diese müssen bis zu einem bestimmten Zeitpunkt abgelegt sein.
In den meisten Fällen bestehen Vorleistungen allerdings aus der Abgabe einer bestimmten Anzahl an Übungsaufgaben.

\minisec{Sprachausbildung}\label{sec:sprachausbildung}

Es werden von der TUD Kurse für fast alle möglichen (und unmöglichen) Sprachen angeboten.
Zu diesem Zweck gibt es zwei Zentren für die Sprachausbildung: Das \glqq Lehrzentrum Sprachen und Kulturen\grqq\ (LSK) und \glqq TUD Institute of Advanced Studies\grqq\ (TUDIAS).
Das Sprachangebot der beiden Einrichtungen ähnelt sich sehr stark. Allerdings ist die Sprachausbildung am TUDIAS im Gegensatz zum LSK kostenpflichtig.
Du hast für diverse Sprachkurse ein Budget an Semesterwochenstunden (insgesamt 10 SWS), die du wie du willst ausgeben kannst.
Für dein Studium zum Bachelor der (Medien-)Informatik sind Sprachkurse generell optional, aber auf jeden Fall empfehlenswert.
Für Diplomstudenten sind 4 SWS Englisch (also 2 Semester) im Laufe ihres Studiums Pflicht.
Studierst du allerdings Bachelor Informatik und möchtest danach mit dem Master Informatik an der TU weitermachen, wirst du genauso 4 SWS Englisch nachweisen müssen.
Die Einschreibung für einen Sprachkurs erfolgt online \link{http://sprachausbildung.tu-dresden.de} mit deinem ZIH-Login.
Sobald die Kurse freigeschaltet sind, solltest du dich jedoch stark beeilen, die beliebten Kurse sind meist innerhalb weniger Minuten voll.
Weitere Infos findest du unter \link{https://tu-dresden.de/lsk/lskonline} und \link{http://www.tudias.de/de/Sprachschule.html}.
