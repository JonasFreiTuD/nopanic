\addchap{Flüchtlinge in Dresden}

Wenn du aus Dresden kommst, wirst du es bereits mitbekommen haben, aber für alle, die neu in unserer Stadt sind: Wie eine Vielzahl anderer Städte in Deutschland (und ganz Europa), ist auch Dresden mit seinen 500.000 Einwohnern Anlaufpunkt für viele geflüchtete Menschen geworden. Die aktuelle politische Lage in Europa und die sächsische Flüchtlingspolitik stellen dabei die Stadt, unzählige freiwillige Helfer, offizielle Hilfsorganisationen wie das DRK, und letztendlich unsere ganze Gesellschaft vor eine große Herausforderung. 

Die meisten der geflüchteten Menschen sind mit ihrer Ankunft in Deutschland und Dresden dem Krieg und der Verfolgung in ihrem Heimatland entkommen und nach teilweise mehrjähriger Flucht nun hier in Sicherheit. In Dresden gibt es mehrere Erstaufnahmestellen, die hunderten Flüchtlingen vorläufig Unterkunft geben, bis ihnen eine dezentrale Unterkunft in oder um Dresden vermittelt wird. Eine stellt die Zeltstadt an der Bremer Straße in Dresden Friedrichstadt dar, in der über 1000 Menschen untergebracht sind.

Auch im USZ \link{http://tu-dresden.de/usz}, in den Sporthallen an der Nöthnitzer Straße direkt neben dem APB[citation needed], ist im August diesen Jahres eine Erstaufnahmestelle für mehr als 600 Menschen, darunter etwa 90 Kinder, eingerichtet worden. \link{http://www.dnn-online.de/dresden/web/dresden-nachrichten/detail/-/specific/Ab-Montag-sollen-rund-600-Asylsuchende-in-Dresdner-TU-Turnhallen-untergebracht-werden-571191965}  Beide Unterkünfte werden vom DRK \link{http://drksachsen.de/faq-fluechtlingsarbeit/} betreut.

Die Umnutzung des USZ hat für uns Studierende den Nebeneffekt, dass kommendes Semester in diesen Hallen keine Sportangebote stattfinden werden – nach Ausweichlösungen wird aber gesucht, alle Infos dazu findet ihr dann auf den Seiten des USZ und das Rektorat informiert regelmäßig über Neuigkeiten und Änderungen, die die Unterbringung im USZ betreffen. 

Außerdem wurden die Schließzeiten des APB verändert. Bisher war die Fakultät immer offen, jetzt ist das nur noch Montag bis Freitag von 6 bis 19 Uhr der Fall. Das heißt aber nicht, dass ihr zu den anderen Zeiten nicht rein oder raus kommt. An der rechten inneren Tür befindet sich eine Klingel - gegen Vorzeigen des Studentenausweises lässt euch der Wachmann jederzeit rein - und raus kommt man ohnehin immer.

Ohne das ehrenamtliche Engagement vieler Bürger Dresdens und vieler Studierender wäre die Unterbringung der bisher etwa 1200 Geflüchteten, die dieses Jahr neu in Dresden angekommen sind, nicht möglich gewesen. Momentan leben etwa 2600 Asylsuchende in Dresden und es werden bis Ende des Jahres noch gut 2000 Neuankömmlinge erwartet. \link{https://www.dresden.de/media/pdf/sozialamt/Zahlen-und-Prognosen-Asyl.pdf} (Zahlen vom 31. Juli 2015) 

Nach wie vor ist viel Arbeit zu tun, da auch weiterhin Flüchtlinge nach Dresden kommen werden, und die Hilfsorganisationen leider überfordert sind. Es gibt in Dresden sehr viele Initiativen und Gruppen, die sich für Flüchtlinge, Asylsuchende und Migranten einsetzen. Das im April 2015 gegründete Netzwerk "Dresden für Alle" \link{https://www.facebook.com/dresdenfueralle; http://dresdenfueralle.de/blog/de/}, bietet eine gute Anlaufstelle, wenn man sich in Dresden engagieren möchte. In diesem Netzwerk sind über 80 Vereine und Organisationen Dresdens vertreten. Außerdem gibt es in einigen Stadtteilen Willkommensbündnisse, die stets freiwillige Helfer suchen.

Weiterhin bietet die interaktive Übersichtskarte Afeefa \link{http://afeefa.de} einen Eindruck über die Initiativenlandschaft der Stadt. Auch direkt an der Uni kannst du dich als Helfer melden und z.B. im USZ beim DRK mitwirken. Dazu wende dich an mithilfe@tu-dresden.de mit folgenden Informationen:

Name, Vorname, Geburtsdatum, ab wann einsetzbar, für wie lange einsetzbar, welche Qualifikationen und speziellen Fähigkeiten, Mailadresse, Tel.- oder Handynummer, Sprachen, Einwilligung zur Weitergabe dieser Angaben ans DRK.

Außerdem existiert zur Koordination der studentischen Hilfe in der Einrichtung im USZ eine Facebookgruppe \link{https://www.facebook.com/groups/helferfureaedd2}.