\addchap{Wichtige Gebäude auf dem Campus}

Obwohl sich dies der Eine oder Andere vielleicht wünschen würde, wirst du nicht dein Ganzes Studium nur im APB absolvieren können.
Stattdessen gibt es noch einige andere wichtige Gebäude, die in deinem Studienalltag eine Rolle spielen werden.

\minisec{SLUB}

Die Sächsische Landesbibliothek – Staats- und Universitätsbibliothek Dresden, kurz \emph{SLUB} ist das, was an anderen Universitäten einfach Bib heißt.
Mit einer Auswahl von über 12 Millionen Bestandseinheiten bestehend aus Büchern, Magazinen, Filmenetc, etc. ist sie eine der größten Bibliotheken und vor Allem Universitätsbibliotheken Deutschlands.
Wer denkt, für ein erfolgreiches Informatikstudium nie ein Bucht in die Hand nehmen zu müssen, liegt jedoch komplett richtig, denn dafür bietet die SLUB auch eine umfangreiche Auswahl online verfügbarer Ressourcen, die ihr euch als Studierender kostenfrei herunterladen könnt.
Doch auch für notorische Nicht-Leser ist die SLUB ein beliebter Aufenthaltsort unter Studierenden, besonders wegen der vielen ruhigen Arbeitsplätze, die im Gebäude zur Verfügung stehen.
Wollt ihr mit Anderen gemeinsam lernen, gibt es dafür einen weiträumigen Eingangsbereich mit Gruppentischen.
Darüber hinaus können private Gruppenräume reserviert werden. Ein schönes Café, in dem per Mensakarte bezahlt werden kann, rundet das Ganze ab.
Vom APB aus befindet sich die Bibliothek jedoch am anderen Ende des Cmapus, was aber für Studienanfänger oft kein Problem darstellt.

\minisec{HSZ}

Denn gerade die Grundlagenvorlesungen finden nicht im APB, sonder im Hörsalzentrum (\emph{HSZ}) statt.
In diesem im Vergleich zum APB doch sehr eintönigem Gebäude werdet ihr viele Vorlsungen zu den Grundlagen der Informatik und beim Pendeln zwischen APB und HSZ den einen oder anderen Kilometer ansammeln.
Das HSZ umfasst vier Vorlesungsäle, unter Anderem das Audimax, den größten Hörsaal der Universität und das größte Auditorium Sachsens.
Zusätzlich finden sich noch einige Seminarräume, in welchen Übungen stattfinden können. Vor dem "zentralen Kubus" findet sich meist ein Mensawagen für den kleinen Hunger zwischendurch.
Auf der anderen Seite des Gebäudes ist die schöne Wiese des HSZ Veranstaltungsort für Feste oder Messen.
Zudem findet ihr dort den Grillcube, der euch für die Pausen zwischen Vorlesungen mit Burgern versorgt.

\minisec{Mensen}

Wer sich nicht mit Fast- und Fingerfood zufriedengeben will, muss dies zum Glück auch nicht, denn das Studentenwerk Dresden betreibt neben dem Grillcube 17 weitere Mensen.
Egal in welchem Gebäude der Universität ihr euch befindet, in der Nähe wird eine Mensa, oder zumindest ein Café zu finden sein.
Besonders die Alte Mensa in der Nähe der Fakultät UND des HSZs ist dabei besonders interessant und gleichzeitig die größte Mensa.
Hier findet ihr eine Vielzahl an Hauptgerichten, Salaten und Nachspeisen - natürlich jeden Tag etwas anderes.
Die aktuellen Gerichte für alle Mensen findest du auf der Seite des Studierendenwerks \link{https://www.studentenwerk-dresden.de/mensen/speiseplan/}.
Für Freunde des späten Frühstücks, oder Studierende mit gestörtem Schlafrhythmus findet sich in der Alten Mensa MO-DO bis 20 Uhr ein abgespecktes, aber warmes, Abendangebot.
Die Mensa kann zu Stoßzeiten doch schon ziemlich voll werden, weswegen sich Zeiten, die NICHT direkt nach Ende einer Doppelstunde sind, besonders anbieten.
Auch am Wochenende könnt ihr dem Kochen entkommen, denn der Siedepunkt hat als einzige Mensa auch an Samstagen und Sonntagen geöffnet.
Gerade nach einer produktiven Lerneinheit bietet sich die Siedepunktmensa an, befindet sie sich doch direkt gegenüber der SLUB und des..

\minisec{Seminargebäude(s)}

In diesem Bau, an der besonders schönen Fassade erkennbar, finden die Sprachkurse des LSK statt.
Für alle, die über einen einfachen Sprachkurs hinausgehen wollen, werden passend zum Namen Seminare zur Kultur und Politik ausgewählter Länder und Regionen angeboten.
Weitere Informationen findet ihr auf der Website des LSK.
Das Gebäude ist ca. 20 Minuten zu Fuß vom APB entfernt und besteht aus zwei Gebäudeteilen.
