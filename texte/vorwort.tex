\textbf{Danke an ...}
\\

\begin{tabular}{l l l} 

Jakob Blume & Ulrich Huber & Mike Pohl\\
Anika Borchmann & Sönke Huster & Axel Reinicke\\
Anna Brauer & Christian Kabelitz & Franziska Ressel\\
Thomas Bruhn & Christoph Kepler& Marcel Rösler\\
Markus Damm & Maximilian Kindt & Simon Rother\\
Jasmin Delling & Clemens Köhler & Marc Satkowski\\
Felix Döring & Kilian Költzsch & Michael Schneider\\
Lars Engeln & Max Korn & Sebastian Schrader\\
Niklas Fallik & Ben Kosmann & Franz-Wilhelm Schumann\\
Paul Genssler & Alexandra Krien & Lara von Schumann\\
Bettina Groschopp & Raphael Lais & Max Staff\\
Anita Grützner & Dirk Legler & Patrick Stiller\\
Lukas Haack & Adrian Lieber & Manuel Thieme\\
Sebastian Hahn & Katja Linnemann & Lucas Vogel\\
Simon Hanisch & Ian Alexander List & Sebastian Vogt\\
Thomas Hauptvogel & Sebastian Mielke & Tilo Werdin\\
Frank Hedecke & Richard Mörbitz & Jens Wettlaufer\\
Joschka Heinrich & Michael Nix & Jan-Erik Wieczorek\\
Philipp Heisig & Dominik Olwig & Felix Wittwer\\
Robin Herrmann & Sascha Peukert & Lucas Woltmann\\

\end{tabular}

\addchap{Vorwort}

Hallo Uniwelt!

heißt es nun für dich als frisch Immatrikulierter, Ersti, an der TU Dresden. 
Endlich kannst du nach Jahren der Knechtschaft selbst über dich und dein Leben bestimmen. 
Wie du mit dieser Freiheit und der daraus folgenden Verantwortung zurecht kommst, lernst du schnell. 
Damit dir der Übergang leichter fällt, veranstaltet dein Fachschaftsrat die Erstsemestereinführung (ESE). 
Eine Woche lang gibt es neben Spiel und Spaß sehr viel Informatives zum Studium sowie zum Unileben allgemein. 
Dieses Heft ist ein nützlicher Ratgeber und nicht vergessen: 
\textbf{NO PANIC!} (Aus historischen Gründen hier nicht das grammatikalisch korrekte \glqq don't panic\grqq.)

Du wirst auch entdecken, dass Uni mehr ist als nur studieren. 
Neben allerlei Erstsemesterpartys gibt es noch mehr zu erleben. 
Gerade prägend für die Dresdner Hochschulkultur sind die 15 Studentenclubs, wie z.B. das Count Down in der Johannstadt. 
In der Neustadt laden ebenso viele Kneipen und Clubs zu langen Nächten ein. 
Einmal im Jahr entlädt sich dieses alternative Flair während der BRN (Bunte Republik Neustadt). 
Und wem das alles viel zu hektisch ist: der fläze sich gemütlich in ein Sofa des ASCII, dem Studentencafé der Fakultät. 
Dort kann man gut bei Kaffee und Club Mate (empfehlenswert auch die lokale Kolle-Mate!) entspannen oder versuchen, doch etwas für die Uni zu tun.

Engagement wird an der TU Dresden groß geschrieben. 
Es gibt viele Hochschulgruppen, die um eure Mitarbeit buhlen. 
Darunter einige politische, wie auch technische, journalistische, künstlerische und und und. Mehr dazu findest du auf der Seite des Studentenrates (StuRa).

Dieses Heft enthält übrigens auch eine Vielzahl von Links zu relevanten Unterseiten auf den Seiten des Fachschaftsrates (FSR), der Uni und anderen. 
Diese sind mit Zahlen wie dieser hier \link{https://html5zombo.com} versehen und ganz am Ende des Heftes gelistet. Ebenfalls kannst du auch direkt unter \url{ese.ifsr.de/2015/<Zahl>} auf die verlinkte Seite weitergeleitet werden.

\textbf{Zu guter Letzt: Wir (deine ESE-Tutoren) wünschen dir viel Erfolg und auch ordentlich Spaß beim Studium!}
