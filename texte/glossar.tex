\twocolumn

\addchap{Glossar}

\textbf{AG DSN} \\
Die AG Dresdner Studentennetz kümmert sich um das Internet in einigen Wohnheimen.
Mithelfer werden laufend gesucht.

\textbf{Anmelden} \\
Alle, die in Dresden heimisch geworden sind, sollten nicht vergessen, sich beim Ortsamt des jeweiligen Stadtbezirkes innerhalb von zwei Wochen anzumelden.

\textbf{APB} \\
Steht für Andreas-Pfitzmann-Bau und ist seit Mitte 2014 offiziell der Name der Fakultät Informatik, deren Kürzel vorher INF war. Ihr werdet sicherlich auf beide Kürzel stoßen, die Änderung ist noch sehr frisch.

\textbf{AQuA} \\
Abkürzung für Allgemeine Qualifikation.
Ist ein Bestandteil deines Studiums.
Genaueres:
Siehe Prüfungs- und Studienordnung.

\textbf{Assistent} \\
Wissenschaftlicher Mitarbeiter am Lehrstuhl, meist Doktor.
Leitet oft Übungen oder Seminare.

\textbf{Auslandsstudium} \\
Etwas, das sich im Lebenslauf immer ganz gut macht, von der Erfahrung und möglicherweise guten Bräune ganz abgesehen.
Nähere Informationen gibt es entweder bei uns im FSR oder im Akademischen Auslandsamt.

\textbf{Bachelor} \\
Die neuen bundesweit eingeführten Abschlüsse.
Merkmale sind ein im Vergleich zum Diplom kürzeres Studium und die Möglichkeit, aufbauend einen Master zu erwerben.

\textbf{BAföG} \\
Zum Thema BAföG gibt es sowohl beim StuRa als auch im Studentenwerk Infomaterial und Anträge.
Beantragt wird BAföG beim BAföG-Amt im Studentenwerk, Fritz-Löffler-Str. 18.
Kümmere dich so schnell wie möglich darum, da frühestens ab dem Antragsmonat gezahlt wird.

\textbf{Belegen} \\
Das Hören einer Vorlesung wird auch als Belegen bezeichnet.
Die im Semester gehörten Vorlesungen müssen in den Belegbogen auf der Rückseite des Studienbuchblattes, das dir mit dem Studentenausweis zugeschickt wurde, eingetragen werden.
Dieses solltest du im Studienbuch abheften.
% macht das überhaupt irgendwer? :D

\textbf{Beurlaubung} \\
Auf Antrag gewährt die Uni zwei freie Urlaubssemester.
Nutz diese Möglichkeit, falls du mal ein Semester frei nehmen willst/musst, damit dir dieses Semester nicht als Fachsemester angerechnet wird.
Achte jedoch auf Bestimmungen zur Höchststudiendauer vor allem zum BAföG.

\textbf{Bibliothek} \\
Primär von Interesse ist für dich die Universitätsbibliothek (SLUB), die du kostenlos nutzen kannst.
Abgesehen davon stehen dir natürlich auch die städtischen Bibliotheken Dresdens zu Verfügung.
Allerdings gibt es für diese eine Jahresgebühr von 12 EUR.

\textbf{Bücher} \\
Es ist ratsam, nicht direkt zum ersten Semester einen Stapel Bücher zu kaufen.
Besser ist es, sich bei höheren Semestern vorher zu erkundigen, welche Literatur ratsam ist.
Außerdem sollte man sich die Bücher, die von Professoren vorgeschlagen werden, zunächst erstmal in der Bibliothek anschauen.
Angebote für gebrauchte Bücher findest du unter anderem in den Campuszeitungen.

\textbf{Campus} \\
Kerngelände der Uni.

\textbf{Campuszeitung} \\
Die zwei Dresdner Campuszeitungen \textit{ad-rem} und \textit{CAZ} erscheinen ein- bzw. zweiwöchentlich.

\textbf{Club Mate} \\
Das ultimative Kultgetränk unter Hackern dieser Welt und im ASCII erhältlich.
Positiver Nebeneffekt nach dem Genuss von Club Mate ist, dass der hohe Koffeingehalt munter macht/hält.
(Nicht mehr ganz so Geheim-)Tipp:
Auch mal die Dresdner Kolle-Mate im ASCII probieren.

\textbf{Creditpoints} \\
Sammelst du mit dem Bestehen von Modulen.
Die Anzahl gibt an wieviel Zeit du aufgewendet hast, bzw. haben sollst.

\textbf{DAAD (Deutscher Akademischer Austauschdienst)} \\
Deutschlandweite Anlaufstelle für das Auslandsstudium.

\textbf{Dekan} \\
Der Dekan leitet und vertritt die Fakultät und führt die Beschlüsse des Fakultätsrates aus.
Der gegenwärtige Dekan ist Prof. Baader.

\textbf{dies academicus} \\
Am "akademischen Tag" finden anstelle der Vorlesungen und Übungen andere Veranstaltungen statt.
Er dient dazu, den Studenten die Möglichkeit zu geben, einmal einen Blick in andere Fachbereiche zu werfen.

\textbf{Diplom} \\
Alternativer Studienabschluss zum Bachelor.
Im Wintersemester 2010 wurde ein neuer, modularisierter Diplomstudiengang (nur Informatik und Informationssystemtechnik) an unserer Fakultät eingeführt.
Im Gegensatz zum Bachelor bietet dieser ein Nebenfach und ein Praktikumssemester.
Das Diplom berechtigt wie ein Master zur Promotion zum Doktor.

\textbf{DrePunct} \\
Bibliothek am Zelleschen Weg 17 (gegenüber der SLUB), die unter anderem die Bücher des Fachbereichs Informatik beinhaltet.

\textbf{Emeal} \\
Der Emeal (auch Mensakarte) wird gebraucht, um in den meisten Mensen Essen zu bekommen.
Er ist gegen eine Kaution von 5 EUR und Vorlage der Emeal-Bescheinigung sowie des Personalausweises an den Kassen der Mensen erhältlich.
Zu Beginn des jeweils nächsten Semesters muss der Emeal verlängert werden.
Im Rahmen der ESE wird die Mensakarte aber auch direkt ausgeteilt.

\textbf{Erasmus} \\
Eine europaweite Initiative zum Studentenaustausch.
Siehe auch Auslandsstudium.

\textbf{EVA} \\
Lehrevaluation, gegen Ende des Semesters füllst du in jeder Vorlesung einen Fragebogen aus, um den Dozenten, die Vorlesung und die Übungsleiter zu bewerten.

\textbf{Exmatrikulation} \\
Beim Austritt aus der Hochschule (Studienende/-abbruch, Wechsel der Hochschule) muss man sich exmatrikulieren.
Zwangsweise geschieht dies, wenn man die Höchststudiendauer überschreitet oder vergisst, sich rückzumelden oder notwendige Prüfungen endgültig nicht bestanden hat.

\textbf{Fachschaft} \\
Alle Studenten einer Fakultät. Also auch du.

\textbf{Fachschaftsrat} \\
Gewählte studentische Vertreter einer Fachschaft.
Deine studentischen Vertreter findest du im Raum E017.
Der Fachschaftsrat freut sich auch immer über Studenten, die mal vorbeischauen und über Probleme oder Anregungen berichten.

\textbf{Fachschaftsratsitzung} \\
Findet einmal wöchentlich im Fachschaftsrat statt.
Hier werden Aktionen geplant, Angelegenheiten der Fakultät diskutiert und vieles mehr.
Jeder ist dazu herzlich eingeladen!
Termine und Sitzungsprotokolle gibt es auf der FSR-Homepage.
Derzeit:
Jeden Montag 18.30 Uhr im großen Ratssaal (APB/1004).

\textbf{Fakultät} \\
In Fakultäten werden verschiedene Fachrichtungen zu einer Lehr- und Verwaltungseinheit zusammengeschlossen (z.B. Fakultät Informatik, Philosophische Fakultät, etc.).

\textbf{FRZ (ZIH)} \\
Das Rechenzentrum in der Informatikfakultät wurde früher von dieser betrieben.
Heute gehört es mit zum ZIH.
Der Rechnerpool bietet dir Gelegenheit, deine Projekte innerhalb der Fakultät zu bearbeiten.
Vorlesungsskripte und Übungsaufgaben einsehen und ausdrucken gehört zu den häufigeren Nutzungen der Rechner.

\textbf{Hochschulsport} \\
Siehe USZ.

\textbf{Immatrikulationsamt} \\
Zuständig für Aktivitäten wie Immatrikulation, Exmatrikulation und Rückmeldung.
Zu finden im Bürohaus Strehlenerstr. 24, 6. Etage und im Netz \link.

\textbf{I'm So Meta Even This Acronym}
%oh yeah

\textbf{INF} \\
siehe APB.

\textbf{Integrale} \\
Kommentiertes Vorlesungsverzeichnis, in dem alle studium generale Veranstaltungen zu finden sind.

\textbf{jExam} \\
Online-Plattform für Studenten.
Hier kannst du dich für Übungen, Seminare, Praktika, Prüfungen, etc. einschreiben und deine Prüfungsergebnisse abrufen.
Zur Einschreibung in der ESE Woche richtest du deinen Account ein und wir zeigen dir direkt, wie du dich einschreiben kannst.

\textbf{Kino} \\
In Dresden gibt es mehrere Kinos, sowohl wahre Paläste für die unbeschwerte Popcornunterhaltung, als auch kleinere Programmkinos wie beispielsweise das Kino in der Fabrik und Thalia.
Es sei hier auf den spiritus rector verwiesen, in dem alle Kinos aufgeführt sind.

\textbf{Klausur} \\
Schriftliche Prüfung zu einer Vorlesung, meist am Ende des Semesters.
Auf der Seite des FSR sind viele Klausuren vergangener Jahre erhältlich (dieses Archiv ist nur aus dem Uninetz erreichbar).

\textbf{Kopieren} \\
An vielen Stellen der Uni stehen Kopierer.
Um sie zu benutzen, braucht man eine Kopierkarte.
Die Karten der Firma Ricoh sind im StuRa (Zimmer 1 bzw. 4) und an verschiedenen Kartenautomaten, die über den Campus verteilt sind, gegen einen Pfand von 5 EUR erhältlich.
Will man mit den Karten auch drucken, sollte man dies vorher angeben.
Bei Bedarf lässt sich diese dann an den entsprechenden Automaten aufladen.
Eine Kopie kostet 5 Cent.
Zusätzlich hast du die Möglichkeit, im Büro des FSR zu kopieren.
Zu guter Letzt hat auch die SLUB ein von der Firma Acribit betriebenes Kopier-/Drucksystem, natürlich auch mit einer eigenen Karte.
Außerdem gibt es auf dem Campus verteilt noch etliche Copyshops.

\textbf{Krankenversicherung} \\
Ab dem 25. Lebensjahr musst du eine eigene abschließen, bis dahin bist du meist über die Familienversicherung deiner Eltern mit abgesichert.
Informiere dich bestenfalls direkt bei deiner Krankenkasse zu diesem Thema.

\textbf{Kryptografie} \\
Mathe, die deine Kommunikation beschützt.
Such' mal nach GnuPG, signiert und verschlüsselt deine E-Mails.

\textbf{Leistungsnachweis, Schein} \\
Muss in einigen Fächern erbracht werden, um zu bestimmten Prüfungen zugelassen zu werden.
Im Gegensatz zu den Prüfungen ist er beliebig oft wiederholbar und meist unbenotet.
Das ist jedoch kein Freibrief zum Durchfallen, da man die Scheine für Klausuren oder die Bachelorprüfung benötigt.

\textbf{LSK} \\
Lehrzentrum Sprachen und Kulturen, s. Seite \pageref{sec:sprachausbildung}.

\textbf{Matrikelnummer} \\
Die Nummer, unter der du an der Uni als Student geführt wirst.
Steht auf deinem Studentenausweis.
Du brauchst sie z.B. bei Klausuren und Prüfungen.
Es ist deswegen günstig, sie auswendig zu wissen bzw. den Studentenausweis immer dabei zu haben.
Letzteres lohnt sich sowieso, da er auch dein Semesterticket ist.
Matrikelnummer bitte nicht mit der s-Nummer verwechseln.

\textbf{Mensa} \\
Es gibt mehrere Mensen auf dem Campus und an den verschiedenen ausgelagerten Fakultäten.
Im Zuge der allgemeinen Technisierung ist in der Mensa ein bargeldloses Zahlungssystem (Emeal) eingeführt worden.
Wo sich welche Mensa befindet und was es an bestimmten Tagen dort Leckeres zu Essen gibt, kann man auf der Seite des Studentenwerkes in Erfahrung bringen.
Für Smartphones gibt es auch etliche mobile Apps.

\textbf{N.N. (nomen nominandus)} \\
Zu Deutsch:
"(noch) zu nennender Name".
Bedeutet:
der Dozent steht noch nicht fest.

\textbf{No Panic} \\
Dieses Heft.
Ein Eigenname aus historischen Gründen und kein falsches Englisch.

\textbf{Prüfungen} \\
Irgendwann muss da jeder ran.
Hierüber sollte man sich genauestens in der Prüfungsordnung informieren.
Prüfungen können nur begrenzt wiederholt werden.
Wichtig ist natürlich auch die Anmeldung zur Prüfung, diese auf keinen Fall vergessen!
Auch daran denken zu jeder Prüfung den Perso und den Studentenausweis dabei haben!
Prüfungen zu schieben ist auch nur eine begrenzt gute Idee, das holt einen schnell wieder ein.

\textbf{Prüfungsamt} \\
Um zu bestimmten Prüfungen zugelassen zu werden, muss man sich beim Prüfungsamt dazu anmelden.
Eventuell muss man auch Scheine, die für die jeweilige Prüfung Voraussetzung sind, vorzeigen.
Weiterhin kann man hier auch Prüfungsergebnisse erfahren und sich für's Nebenfach einschreiben.

\textbf{Prüfungsordnung} \\
Dort erfährst du, welche Prüfungen und Leistungsnachweise für die Bachelorprüfung benötigt werden und welche Fristen einzuhalten sind.
Diese sollte unbedingt gelesen werden, damit man zumindest weiß, warum man irgendwann plötzlich exmatrikuliert wurde.

\textbf{Prüfungszeit} \\
In den Wochen nach den Vorlesungen wirst du wahrhaftig geprüft.
Prüfungen sollten sechs Wochen vor der Prüfungsperiode im Termin feststehen.
Zu einer Prüfung muss man sich per jExam anmelden.
Abmeldungen (Rücktritte) sind unter bestimmten Voraussetzungen ebenfalls über jExam möglich.

\textbf{Rechtsberatung} \\
Eine kostenlose Rechtsberatung bietet dir der StuRa (Do 13-14 Uhr, 14-tägig) und der Justiziar des Studentenwerkes.
% stimmt noch

\textbf{Rektor} \\
Leitet und vertritt die Universität.
Derzeit Prof. Hans Müller-Steinhagen.

\textbf{Rekursion} \\
Siehe Rekursion.

\textbf{Rückmeldung} \\
Jeder Student, der im darauffolgenden Semester weiter an der Uni studieren möchte, muss sich im angegeben Zeitraum rückmelden.
Die Rückmeldung erfolgt durch fristgemäßes Überweisen des Semesterbeitrags.
Das Formular dafür befindet sich auch immer auf dem Semesterbogen.
Die Höhe des Semesterbeitrags wird auf den Webseiten des Immatrikulationsamts bekanntgegeben.

\textbf{Rundfunkgebühren} \\
Studenten, die nicht zu Hause wohnen, müssen ihren Haushalt anmelden.
Für manche Studenten (z.B. BAföG-Empfänger) besteht jedoch die Möglichkeit, sich von der Gebührenpflicht befreien zu lassen.
Dies ist direkt bei der Rundfunkzentrale zu beantragen.
Bedenke auch, dass du rückwirkend mit deinem Einzugsdatum zur Kasse gebeten werden kannst, eine verspätete Anmeldung bringt also keinen Vorteil.

\textbf{Schein} \\
Siehe Leistungsnachweis.

\textbf{Semesterticket} \\
Wird automatisch mit der Überweisung des Semesterbeitrags bezahlt.
Dein Studentenausweis in Verbindung mit einem gültigen Personalausweis gilt als Fahrschein und ist nicht übertragbar.
Dein Semesterticket gilt auch für den Regionalbahnverkehr in ganz Sachsen.

\textbf{SHKs} \\
Studentische Hilfskräfte werden von den Lehrstühlen als Tutoren oder für wissenschaftliche Hilfstätigkeiten eingestellt.

\textbf{Skript} \\
Oft veröffentlicht der Dozent einer Vorlesung ein eigenes Skript, das dann im Netz öffentlich zugänglich ist und ausgedruckt werden kann.
Diese Skripte sind jedoch nur als Gerüst einer Vorlesung anzusehen und reichen nicht für ein selbständiges Eigenstudium aus.
Damit wollen die Professoren verhindern, dass niemand mehr in den Vorlesungen auftaucht.

\textbf{SLUB} \\
Das Hauptgebäude der Sächsischen Landes-, Staats- und Universitätsbibliothek befindet sich am Zelleschen Weg 18 und ist nicht nur wegen seines schönen, ruhigen Lesesaals immer einen Besuch wert.
Dazu gibt es einige Zweigstellen wie den DrePunct.
Zum Ausleihen von Büchern benötigst du einen Bibliothekausweis, den man jederzeit in der Hauptbibliothek beantragen kann.

\textbf{spiritus rector} \\
Der \glqq leitende Geist\grqq , ein unentbehrliches Heftchen, jedes Jahr von einigen Enthusiasten im StuRa hergestellt.
In ihm kann man u.a. sämtliche Adressen von Kneipen oder Fachschaftsräten finden.
Erältlich beim FSR oder direkt beim StuRa.

\textbf{STAV} \\
Die studentische Arbeitsvermittlung bietet eine Liste von aktuellen Jobs an.
Findet man in der StuRa-Baracke oder auf deren Webseite.

\textbf{Studentenrat (StuRa)} \\
Er vertritt die studentischen Interessen gegenüber der Universität und der Politik und kümmert sich unter anderem um die Verhandlung deines Semestertickets oder um gravierende Probleme mit dem Studentenwerk oder anderen Institutionen.
Außerdem bietet er auch Beratung bei studienrelevanten Problemen (BAföG, etc.) an.
In der StuRa-Baracke befinden sich neben dem Servicebüro des StuRas auch die Büros von STAV und Integrale.

\textbf{Studentenwerk} \\
Fritz-Löffler-Str. 18.
Das Studentenwerk ist zuständig für die Mensen, Studentenwohnheime, BAföG, Beratungen, Wohnungsvermittlung, etc.

\textbf{Studienbuch} \\
In das Studienbuch musst du deine ausgefüllten Studienbuchblätter zusammen mit erlangten Scheinen abheften.

\textbf{Studienordnung} \\
Die Studienordnung legt einen Rahmen für den Ablauf eines Studiums fest, z.B. welche Vorlesungen gehört werden sollten.
Studienordnungen kannst du beim Prüfungsamt, der Studienberatung oder beim FSR bekommen.
Außerdem solltest du im Laufe der ESE zusammen mit dieser No Panic eine erhalten haben.
Unbedingt mal lesen, denn sie enthält deine Rechte und Pflichten.

\textbf{studium generale} \\
Freiwilliges Vorlesungsangebot zum über-den-Tellerrand-schauen.
Siehe Integrale.

\textbf{SWS (Semesterwochenstunden)} \\
Die SWS sind eine Maßeinheit für die Menge von Vorlesungsstunden, die man pro Semester von einer spezifischen Vorlesung besuchen muss.
2 Semesterwochenstunden entsprechen 90 Minuten pro Woche in der Vorlesungszeit.
Wenn man z.B. im zweiten Semester 3 SWS Mathevorlesungen besuchen muss, heißt das, dass man in jeder Woche eine Doppelstunde und zusätzlich alle 2 Wochen noch einmal eine Doppelstunde Mathe zu hören hat.
Durchschnittlich gibt das 3 SWS Vorlesungen in jeder Woche.

\textbf{TUDIAS} \\
TUD Institute of Advanced Studies, s. Seite \pageref{sec:sprachausbildung}.

\textbf{Übungen} \\
Hier wird der Vorlesungsstoff praktiziert.
Es wird von dir als Student erwartet, das Übungsblatt vorher zumindest anzuschauen, um dann die Lösungen zu diskutieren.

\textbf{USZ (Universitätssportzentrum)} \\
Die Universität bietet eine breite Palette von Sportarten zu günstigen Preisen an (normalerweise 15 EUR pro Semester).
Welche Sportarten angeboten werden und wie du dich anmeldest, kannst du auf der Webseite oder im Hochschulsportprospekt, der ab Semesterbeginn überall ausliegt, nachlesen.
Zum Semesterbeginn findet die Einschreibung online statt.
Auf die Termine dafür unbedingt achten, beliebte Kurse sind sehr schnell voll!

\textbf{VL} \\
Vorlesung

\textbf{Wahlen} \\
Gibt es immer im Wintersemester für die Fachschaftsräte der Fakultäten der Uni, die dann Vertreter in den StuRa und in die verschiedenen Gremien entsenden.
Weiterhin können die studentischen Mitglieder des Fakultätsrates gewählt werden.

\textbf{ZIH} \\
Das Zentrum für Informations- und Hochleistungsrechnen.
Es ist zuständig für alles was mit Computern, Logins, E-Mail, WiFi usw. zu tun hat.

\onecolumn
